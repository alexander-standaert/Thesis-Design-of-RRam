\documentclass[master=elt,masteroption=eg]{kulemt}
\setup{title={Ontwerp van een RRAM geheugen voor ingebedde NV toepassingen},
  author={Wouter Diels\and Alexander Standaert},
  promotor={Prof.\,dr.\,ir.\ W. Dehaene},
  assessor={Prof.\,dr.\,ir.\,R. Lauwereins\and Prof.\,dr.\,ir.\, M. Verhelst},
  assistant={ir.\ B.~Baran \and dr.\,ir.\ S.~Cosemans}}
% De volgende \setup mag verwijderd worden als geen fiche gewenst is.
\setup{filingcard,
  translatedtitle={Design of a RRAM memory for embedded NV applications},
  udc=621.3,
  shortabstract={RRAM is een veelbelovende technologie voor het maken van embedded NV geheugens. Deze thesis behandelt het ontwerp van het leescircuit van een 4Mbit RRAM geheugen. Er wordt een geheugenarchitectuur ontworpen dat opgebouwt is uit cellen, branches, local blocks en global blocks. Hierin wordt er een uitgebreide analyse gedaan op het ontwerp en keuze van componenten zoals lastimpedantie en sense amplifier. Nadat omringende logica zoals buffers en decoders ontworpen zijn, wordt er onderzocht wat de optimale geheugenconfiguratie is. Deze optimale configuratie wordt dan onderworpen aan een speed-vdd-test en vergeleken met schakelingen in de literatuur.}}
% Verwijder de "%" op de volgende lijn als je de kaft wil afdrukken
%\setup{coverpageonly}
% Verwijder de "%" op de volgende lijn als je enkel de eerste pagina's wil
% afdrukken en de rest bv. via Word aanmaken.
%\setup{frontpagesonly}

% Kies de fonts voor de gewone tekst, bv. Latin Modern
\setup{font=lm}
\setup{inputenc=utf8}
% Hier kun je dan nog andere pakketten laden of eigen definities voorzien

% Tenslotte wordt hyperref gebruikt voor pdf bestanden.
% Dit mag verwijderd worden voor de af te drukken versie.
\usepackage[pdfusetitle,colorlinks,plainpages=false]{hyperref}
\usepackage{amsmath}
\usepackage{subfig}
\usepackage{afterpage}
\usepackage{tabularx}
\usepackage{hyperref}
\hypersetup{
    citecolor = {blue},
}
%\usepackage{showframe}

%%%%%%%

\begin{document}

\begin{preface}
  Na bijna een jaar aan deze thesis gewerkt te hebben zouden we graag een aantal mensen in het bijzonder willen bedanken voor hun hulp en steun. Ten eerste Prof. Dehaene voor het onderwerp aan te bieden en toe te wijzen aan ons. Ten tweede onze begeleiders Stefan en Burak.\\
  Stefan, hartelijk dank voor de tijd die je wou vrij maken voor ons, het was heel aangenaam om een begeleider te hebben die zo veel kennis en ervaring heeft van het vakgebied.\\
  Burak, thanks for always being available for us and for providing us the resources to get our thesis underway.\\
  Verder zouden we graag Bert DeKnuydt willen bedanken voor zijn hulp met het Condor systeem. Tenslotte zouden we ook elkaar willen bedanken, we waren een goed complemenair team. 
\end{preface}

\tableofcontents*

\begin{abstract}
 Nu het schalen van flash-geheugens op zijn limieten begint te stoten, is er nood aan een alternatief. Met eigenschappen zoals lage voedingsspanning, kleine geheugencel en snelle leessnelheid is RRAM één van de meest belovende kandidaten.\\\\
 Deze thesis behandelt het ontwerp van het leescircuit van een 4Mbit RRAM geheugen. De grootste uitdagingen in het ontwerp van dit resistief geheugen is variabiliteit. Deze variabiliteit is kritisch op twee plaatsen. Ten eerste bij de resistieve deling tussen de geheugencel en een last. Ten tweede bij het latchen van de sense amplifier.\\\\
 Dit werk introduceert verschillende innovaties. Ten eerste een uitgebreide analyse op de keuze van lastimpedantie. Ten tweede het vormen van de referentiespanningsdistributie door middel van parallelle geheugencellen. En tenslotte een analyse over de invloed van een overlappende werking van pass-gates en sense amplifier onder variabiliteit. Verder wordt ook het ontwerp van alle andere belangrijke bouwblokken zoals decoders en buffers besproken.\\\\
 Deze kennis wordt dan gebruikt in het ontwerp van een 4Mbit geheugen in 45nm PTM technologie. Het geheugen maakt gebruik van 512 sense ampliers die telkens gekoppeld zijn aan 2 geheugenmatrices met 32WL en 32BL. Op een voedingspanning van 1V heeft het geheugen een random-access-leessnelheid van 435MHz. Het energieverbruik per leesoperatie is bij deze voedingspanning 0.51pJ. Binnen de afbakening van dit werk presteert de ontworpen schakeling beter dan flash-geheugens gevonden in de literatuur.
\end{abstract}

% Een lijst van figuren en tabellen is optioneel
%\listoffigures
%\listoftables
% Bij een beperkt aantal figuren en tabellen gebruik je liever het volgende:
\listoffiguresandtables
% De lijst van symbolen is eveneens optioneel.
% Deze lijst moet wel manueel aangemaakt worden, bv. als volgt:
\chapter{Lijst van afkortingen en symbolen}
\section*{Afkortingen}

\begin{flushleft}
  \renewcommand{\arraystretch}{1.1}
  \begin{tabularx}{\textwidth}{@{}p{18mm}X@{}}
    BL & Bit Line \\
    CDF & Cumulative Distribution Function \\
    GB & Global Block \\
    HRS & High Resistive State \\
    LB & Local Block \\
    LRS & Low Resistive State \\
    MTJ & Magnetic Tunnel Junction \\
    NoBLpLB & Number of Bit Lines per Local Block \\
    NoGB & Number of Global Blocks \\
    NoWLpB & Number of Word Lines per Branch \\
    PDF & Probability Density Function \\
    PTM & Predictive Technology Model \\
    RAM & Random Access Memory \\
    RRAM & Resistive Random Access Memory \\
    SA & Sense Amplifier \\
    SL & Source Line \\
    WL & Word Line \\
  \end{tabularx}
\end{flushleft}

% Nu begint de eigenlijke tekst
\mainmatter

\chapter{Inleiding}
\label{inleiding}

Vandaag de dag is elektronica niet meer uit het leven weg te denken. Van de smartphone tot het digitaal horloge, van de boordcomputer in de moderne wagen tot de microprocessor in de vaatwasser, overal vind je wel elektronica terug.
Sinds Gordon Moore ongeveer 50 jaar geleden de uitspraak deed dat het aantal transistoren op eenzelfde oppervlakte per twee jaar zou verdubbelen \cite{Moo65}, is de industrie er over het algemeen goed in geslaagd dit te verwezenlijken. Dit leidde tot de snelle en uiterst complexe chips die we vandaag allemaal goedkoop aankopen.

Naarmate de processorkracht groter werd, steeg ook de vraag voor grotere en snellere geheugens om deze processorkracht ook effectief uit te buiten. Static Random Access Memory (SRAM) blijft een populaire keuze voor snelle ingebedde geheugens, maar heeft het nadeel vluchtig te zijn: eenmaal de voedingspanning wordt afgeschakeld, verdwijnt de informatie. Flash-geheugens, door veel mensen gebruikt voor massa-opslag in USB-sticks of SSDs, hebben ook hun weg gevonden tot het ingebedde domein en behoren wel tot de klasse van niet-vluchtige geheugens.
Het blijkt echter bijzonder moeilijk om flash-geheugens verder te verkleinen \cite{Pra10}.

Onderzoek naar nieuwe geheugens is dan ook onontbeerlijk. Zo zijn er al nieuwe nieuwe kandidaten in opmars die hoopgevende tekens geven om te concurreren met (ingebedde) flash-geheugens. MRAMs (Magnetic RAMs) en in het bijzonder STT-RAM (Spin-Transfer Torque) zullen op termijn een belangrijke rol gaan spelen.

Een andere kandidaat is Resistive RAM (RRAM of ReRAM). Daar waar SRAM- en flash-cellen de informatie bevatten via het al dan niet aanwezig zijn van lading, bevat een RRAM-cel informatie door een bepaalde elektrische weerstand aan te nemen. RRAM zou geen problemen hebben om nog even op de klassieke manier mee te schalen en is dus zonder meer een interessante piste om te onderzoeken. Bovendien zou het gefabriceerd kunnen worden met goedkopere processen dan flash-geheugens - bij flash-geheugenfabricatie zijn vaak dure extra maskers vereist terwijl RRAM compatibel is met een standaard CMOS-productieproces.

\section{Doel en afbakening van dit werk}
Dit werk beschrijft het ontwerp van een 4Mbit RRAM-geheugen voor ingebedde toepassingen. De doelstelling is een pareto-optimaal circuit te ontwerpen. De pareto-doelstellingen zijn snelheid, dynamische energie en oppervlakte. Het ontwerp is ook gewapend tegen variabiliteit ontworpen d.w.z. ongecorreleerde gedragsvariaties van componenten. De analyse focust op de leesbewerking, de schrijfbewerking valt buiten het bereik van dit werk. Er worden wel mogelijke oplossingen aangereikt, maar deze werden niet uitdrukkelijk onderzocht. Bij het ontwerp is ook aandacht besteed aan het vermijden van destructieve leescycli.

Voor de leesbewerking wordt het geheugenelement gemodeleerd als een weerstand waarvan de weerstandswaarde afhangt van de celtoestand.Om variabiliteit te onderzoeken, worden Monte Carlo simulaties uitgevoerd waarbij de weerstandswaarde een gaussisch verdeelde variabele is.

Temperatuursvariaties werden niet zorgvuldig onderzocht, maar aangezien dit een globale variabele is en het systeem differentieel werkt, wordt niet verwacht dat de performantie aanzienlijk zal verminderen.

Alle analyses in dit werk zijn uitgevoerd met Spectre simulaties met 45nm PTM transistormodellen. In tabel \ref{tab:properties} zijn technologieparameters te zien, waarvan meermaals gebruik gemaakt wordt doorheen dit werk.

\begin{table}
	\begin{tabular}{ccc}
	\hline
    $A_{\beta n}$ & 2 $\% \mu m$ & $\beta$-Pelgrom constante voor nMOS-transistoren \\
    $A_{\beta p}$ & 1,2 $\% \mu m$ & $\beta$-Pelgrom constante voor pMOS-transistoren \\
    $A_{V_{T} n}$ & 2,82 $mV \mu m$ & VT-Pelgrom constante voor nMOS-transistoren \\
    $A_{V_{T} n}$ & 2,5 $mV \mu m$ & VT-Pelgrom constante voor pMOS-transistoren \\
    $C_{WL}$ & 0,18 fF/cel & WL-capaciteit, stijgt lineair met het aantal cellen eraan \\
    $C_{BL}$ & 0,18 fF/cel & BL-capaciteit, stijgt lineair met het aantal cellen eraan \\
    $C_{inv}$ & 0,35 fF & intrinsieke capaciteit van een CMOS inverter \\
    $\mu_{HRS}$ & 32500 $\Omega$ & verwachtingswaarde van een HRS geheugenelement \\
    $\mu_{LRS}$ & 7500 $\Omega$ & verwachtingswaarde van een LRS geheugenelement \\
    $\sigma_{HRS}$ & 833 $\Omega$ & standaarddeviatie van een HRS geheugenelement \\
    $\sigma_{LRS}$ & 833 $\Omega$ & standaarddeviatie van een LRS geheugenelement \\
    $V_{DD}$ & 1 V & voedingspanning \\
    $V_{SS}$ & 0 V & grondspanning \\
    \hline
  \end{tabular}
  \caption[technologieparameters]{numerieke technologieparameters waarvan gebruik gemaakt is in simulaties}
  \label{tab:properties}
\end{table}


\section{Structuur van de tekst}
In hoofdstuk \ref{cell} wordt de technologie van een RRAM geheugen uiteengezet, alsook diens toepassingen. In hoofdstuk \ref{architecture} wordt het geheugensysteem vanuit vogelperspectief besproken. Er wordt hier ook aangehaald wat de regelbare parameters zijn van de architectuur. Voor een robuuste, snelle en laag-energetische leesoperatie is het belangrijk het geheugenelement te combineren met een zorgvuldig gekozen impedantie, dit wordt onderzocht in hoofdstuk \ref{loadanalysis}. Uiteindelijk worden bits afgeleverd aan de uitgang van het systeem, de sense amplifier zorgt hiervoor en wordt besproken in hoofdstuk \ref{sensamp}.
In de geheugenstructuur worden ook bouwblokken zoals decoders, buffers en passgates om op basis van het opgegeven adres de juiste cel aan te spreken, deze worden beschreven en geanalyseerd in hoofdstuk \ref{periphery}.
De timing van controlesignalen en hoe het systeem te optimaliseren door middel van de architectuurparameters te tunen wordt onderzocht in hoofdstuk \ref{timing-optimization}. Tenslotte wordt een overzicht gegeven van de resultaten van het volledige ontwerp in hoofdstuk \ref{final}.

\chapter{Geheugencel}
\label{cell}
Elk geheugen bestaat uit een verzameling individuele cellen die de informatie bevatten op een manier.
In dit hoofdstuk wordt eerst wat dieper ingegaan om de manier waarom een R-RAM geheugencel informatie bevat en vervolgens hoe deze informatie elektrisch kan worden gebruikt.

\section{Memristor}
Het essentiële element van een R-RAM geheugencel is ontegenspreekbaar de zogenaamde memristor.
De memristor wordt ook wel gezien als de 4\textsuperscript{e} passieve component, naast de weerstand, spoel en condensator.

\subsection{Theoretisch principe}
In 1971 publiceerde Leon Chua een artikel waarin hij opmerkte dat er voor de 4 fundamentele circuitvariabelen (de spanning v, stroom i, lading q en fluxbinding $\lambda$\footnote{$\lambda(t) =  \int^{t}_{-\infty} v(\tau) \, d\tau $, voor een ideale inductantie is dit hetzelfde als magnetische flux: $\lambda = \phi$ }) van 6 mogelijke onderlinge relaties er slechts 5 gekend waren: $q(t) =  \int^{t}_{-\infty} i(\tau) \, d\tau $, $\lambda(t) =  \int^{t}_{-\infty} v(\tau) \, d\tau $, $v(t)=R*i(t)$, $q(t)=C*v(t)$ en $\lambda(t) = L*i(t)$ volgen uit de wetten van Maxwell en uit de definities van de weerstand, spoel en condensator, maar er ontbrak een relatie tussen $\lambda$ en q.\cite{Chu71} Hij suggereerde dat er een 4e nog niet ontdekte passieve 2-pool moest bestaan die dit verband herbergde.
Uit zijn wiskundige berekeningen kwam hij tot de conclusie dat deze component zich ogenblikkelijk als een weerstand zou gedragen, maar dat deze weerstand verandert aan de hand van het verloop van de stroom in de tijd. Gebaseerd op deze conclusie doopte hij deze component de memristor (een contractie van memory en resistor).

\subsection{Fysische werking}
Chua beëindigde zijn artikel met te erkennen dat er op dat moment nog geen fysische memristor was ontdekt, maar dat dit in de toekomst wel kon gebeuren, al dan niet zelfs per ongeluk. Hij gaf zelfs aan dat er misschien al in die tijd materialen met memristorkarakteristieken gebruikt werden, maar dat men hier over keek. Hij zou gelijk krijgen.


\subsection{Toepassingen}


\section{Memristor in een geheugenstructuur}

In dit werk wordt gebruik gemaakt van een \emph{1 Transistor, 1 Resistor} (maar eigenlijk dus een memristor) architectuur, de combinatie van deze twee vormt de geheugencel, maar er zijn nog een paar andere configuraties die zouden toegepast kunnen worden.

\subsection{1T1R}

\subsection{1R}

\subsection{1T1D}



\section{Besluit}


\chapter{Geheugenarchitectuur}
\label{architecture}
De afzonderlijke geheugencellen zullen uiteindelijk samengebracht moeten worden in een geheel.
In dit hoofdstuk zal de algemene structuur besproken worden alsook de vrijheidsgraden die in hoofdstuk \ref{timing-optimization} onderzocht worden voor een optimaal werkend systeem. Ten slotte zullen ook nog de bouwblokken aangekaard worden die meer uitvoerig besproken worden in de volgende hoofdstukken. 

\section{Van transistorniveau tot systeemniveau}
Hoewel het hele systeem op de chip in weze bestaat uit transistoren en passieve componenten, zal dit nog onmogelijk te begrijpen vallen op zulke grote schaal.
Het is daarom aangewezen om meer abstractie te maken en de componenten vanop een hoger niveau te bekijken.

\subsection{Cel}
Zoals besproken in hoofdstuk \ref{cell} zal dit het bouwblok zijn dat het vaakst terug zal te vinden zijn op het geheugensysteem.
De cel bestaat uit een memristor en een transistor. De geheugencel heeft drie terminals: de gate van de transistor, die zal verbonden worden met een wordline, de source van de transistor, die zal verbonden worden met een sourceline en tenslotte de terminal van de memristor, die zal verbonden worden met een bitline.
Een cel wiens memristor zich in een willekeurige resistieve staat bevindt is een datacel, terwijl de memristor van referentiecellen in een voorgeprogrammeerde en dus gekende resistieve staat verkeert.

\subsection{Branch}
In een branch worden er een bepaald aantal datacellen verbonden aan één BL en één SL. Dit aantal wordt \emph{Number of Word Lines per Branch} (NoWLpB) genoemd en is een van de vrijheidsgraden van de geheugenarchitectuur. Naast alle datacellen is er ook nog één referentiecel verbonden aan de BL en SL van de branch.
Elke BL wordt via een pMOS-transistor verbonden met de voedingspanning Vdd en via een nMOS-transistor aan de grondspanning Vss. In dit werk is er enkel een nMOS-transistor die de SL verbindt met Vss.\footnote{In een volledig geheugensysteem zou de SL via een pMOS ook nog verbonden zijn met een niet onderzochte spanningsknoop Vdd\_write. De pMOS zou dan worden aangezet voor schrijfwerking.} De nMOS-transistoren aan BL en SL fungeren als schakelaars, de pMOS-transistor wordt gebruikt als impedantie voor een resistieve spanningsdeling (zie hoofdstuk \ref{loadanalysis}).
Ter illustratie wordt de samenhang tussen cel en branch getoond in figuur \ref{fig:cellbranch}.

\begin{figure}
  \centering
  \includegraphics[scale=0.3]{../fig/hfdstk-architecture-cell-branch.png}
  \caption{Een geheugencel en een branch}
  \label{fig:cellbranch}
\end{figure}

\subsection{Local Block}
Verschillende BLs en SLs worden samengebracht in een Local Block, waarvan de vrijheidsgraad \emph{Number of Bit Lines per Local Block} (NoBLpLB) heet. In een LB bevinden er zich dus NoBLpLB x NoWLpB datacellen en NoBLpLB referentiecellen. Ook zitten er in een Local Block zowel BL- als WL-decoders.
De structuur van een Local Block is geïllustreerd op figuur \ref{fig:LB}.
De uitgangen van de WL-decoder zullen (mits wat buffering tussenin) worden aangesloten aan de data-WLs zelf, die van de BL-decoder zullen een spanningsdeling teweeg brengen op de BLs. De referentie-WL zal via een extern signaal verbonden worden.
Aangezien een LB zowel data- als referentiecellen bevat, gaat een LB twee werkingsmodes hebben: een mode waarbij er één datacel wordt aangesproken en een mode waarbij er een bepaald aantal referentiecellen in parallel wordt aangesproken.

\begin{figure}
  \centering
  \includegraphics[scale=0.3]{../fig/hfdstk-architecture-localblock.png}
  \caption{Een Local Block}
  \label{fig:LB}
\end{figure}

\subsection{Global Block}
\label{globalblock}
Een Global Block bestaat uit twee LBs en een sense amplifier (SA). In het ene LB gaat er een datasignaal geproduceerd worden, in het andere een referentiesignaal (zie figuur \ref{fig:GB}. Vervolgens gaat de SA dit kleine signaalverschil versterken tot een zuivere rail-to-rail output.
Aan de uitgang van het GB verschijnen dan ook de opgevraagde bits.
De laatste architectuurvrijheidsgraad is de \emph{Number of Global Blocks} (NoGB), het totale geheugen bevat dus NoGB x 2 x NoBLpLB x NoWLpB geheugencellen.

\begin{figure}
  \centering
  \includegraphics[scale=0.3]{../fig/hfdstk-architecture-globalblock.png}
  \caption{Een Global Block}
  \label{fig:GB}
\end{figure}

\section{Besluit}



\chapter{Lastimpedantie-analyse}
\label{loadanalysis}
Om een cel uit te lezen wordt er een spanning gegenereerd op de bitline door middel van een spanningsdeling.
Het is dus belangrijk om de 2 impedanties van de spanningsdeler zodanig te kiezen voor optimale snelheid, bitline spanningsverschil en spanningsval over de memristor.
Ook belangrijk is dat deze impedanties robuust zijn tegen variabiliteit.

\section{algemene last eigenschappen en specificaties}\label{sec:simplemodel}
In deze eerste sectie bestuderen we de combinatie van last en memristor cell als een heel simpel model namelijk twee weerstanden in serie (zie figuur \ref{fig:simplemodel}). Dit om aan te tonen dat de weerstands waarde van de last een grote invloed heeft op de het voltageverschil tussen een hoge en lage cel weerstand, bitlijn snelheid en de sensitivity van beide.
Het verschil in bitlijn voltage tussen een hoge en lage cell is van belang voor de toleraties op de referentie voltage en sense amplifier mismatch. In het simple model kan het verschil in bitlijn voltage analitisch berekend worden met de volgende formule:
\begin{equation}
 \Delta V = \frac{R_{HRS}}{R_{last}+R_{HRS}} - \frac{R_{LRS}}{R_{last}+R_{LRS}}
\end{equation} 
Voor constante waarden van $R_{HRS}$ en $R_{LRS}$ zal er een maximum zijn in $ \Delta V$ zoals duidelijk gezien kan worden in figuur \ref{fig:rpiek}. De sensitiviteit van de last weerstand op het spanningsverschill, moet men voorzichtig interpreteren. Op figuur \ref{fig:rpiek} kan gezien worden dat de helling voor de piek stijler is als na de piek. Het is dus beter om een iets grotere weerstand te hebben als een iets te kleine weerstand. Maar als men de weerstand naar transistor afmetingen vertaalt, kan met dit op verschillend manieren realiseren. Een grote weerstand realiseren met een transistor met minimale lengte, zal betekenen dat de breete van de transistor klein moet zijn. Dit zal dan gevoeliger zijn voor mismatch dan een grotere breete van transistor.\\
De snelheid van het opladen van de bitlijn kan in het simple model ook analitisch geschreven worden. De volgende vergelijking stelt de tijd voor waar de bitlijn $99\%$ is opgeladen.

\begin{align}
t = -ln(0.01)*RC\\
R^{-1} = \frac{1}{R_{cell}} + \frac{1}{R_{last}}
\end{align}

Deze tijd zal kleiner worden als de R kleiner word, dit vertaalt zich dan naar een kleine last weerstand.

\begin{figure}[!ht]
\centering
\subfloat[het simpel bitlijn model]{ \includegraphics[width=0.20\textwidth] {../fig/hfdst-last-simplemodel.png} \label{fig:simplemodel}}
\subfloat[Verschil in Bitlijnspanning in functie van lastweerstand]{ \includegraphics[width=0.80\textwidth] {../fig/hfdst-last-rpiek.png} \label{fig:rpiek}}
\caption{}
\end{figure}




\section{evalueren van de last}
Om verschillende lasten met elkaar te kunnen vergelijken, is het belangrijk om hun eigenschappen allemaal op dezelfde manier te bekomen. Figuur \ref{fig:simsetup} geeft de verschillende aspecten van de gebruikte simulatie setup weer. Het test circuit (Figuur \ref{fig:simcircuit}) stelt een bitlijn voor met een capaciteit van 18fF, wat ruiw weg overeenkomt met een bitlijn met 100 cellen op. Aan deze bitlijn zijn een last, een ontlaad transistor en een memristor weerstand aangesloten. De ontlaad transistor is minimaal gehouden. De memristor weerstand kan de volgende waardes hebben: tussen 5k$\Omega$ en 10k$\Omega$ voor de LRS, tussen de 30k$\Omega$ en 35k$\Omega$ voor de HRS. De nominale waardes voor LRS en HRS zijn 7.5k$\Omega$ en 32.5k$\Omega$. Tijdens monte-carlo analyses worden dan deze nominale waardes als gemiddelde van een gausische distributie genomen met $\sigma = 0.833k\Omega$. Aan deze memristor weerstand hangt een select transistor, die ook minimaal gehouden wordt, en de combinatie van deze wordt de geheugen cell genoemt. Aan deze geheugen cell hangt nog een selectlijn transistor met een breedte van 500nm. Deze transistor werd bewust groot gemaakt om de totale weerstand in de onderste tak voornamelijk te laten afhangen van de geheugen cell. Aan deze selectlijn werd er ook een capaciteit van 18fF aan gehangen, deze doet echter niet veel aangezien de select transistor altijd aangelaten wordt. Tenslotte wordt de voedingsspanning altijd op 1V gehouden.\\\\
Figuren \ref{fig:simcontr1} tot \ref{fig:simcontr3} stellen de sequentie voor van alle controlesignalen uit tijden de simulatie. Eerst wordt de bitlijn volledig ontladen (figuur \ref{fig:simcontr1}). Vervolgens is er een interval waar niks gebeurt (figuur \ref{fig:simcontr2}) en tenslotte wordt te last aangesloten en de bitlijn opgeladen (figuur \ref{fig:simcontr3}). De simulatie stopt als de bitlijn volledig opgeladen is.\\

\begin{figure}[!ht]
\centering
\subfloat[Test circuit]{ \includegraphics[width=0.45\textwidth] {../fig/hfdst-last-simsetup.png} \label{fig:simcircuit}}
\subfloat[Controle signalen 1]{ \includegraphics[width=0.45\textwidth] {../fig/hfdst-last-controlsig1.png} \label{fig:simcontr1}}\\
\subfloat[Controle signalen 2]{ \includegraphics[width=0.45\textwidth] {../fig/hfdst-last-controlsig2.png} \label{fig:simcontr2}}
\subfloat[Controle signalen 3]{ \includegraphics[width=0.45\textwidth] {../fig/hfdst-last-controlsig3.png} \label{fig:simcontr3}}
\caption{Test bench voor de last}\label{fig:simsetup}
\end{figure}

Eens een last gesimuleert is wordt het beoordeelt op het vlak van oppervlakte, bitlijn oplaadsnelheid, nominaal bitlijn voltage verschil en spanningsval over de cel. Het oppervlakte wordt berekend op basis van de lengtes en breetes van de last transistoren. De bitlijn oplaadsnelheid is de tijd dat nodig is om de bitlijn $99\%$ op te laden. Het nominale bitlijn voltageverschil is het verschil in bitlijnvoltage tussen een cell in HRS en LRS, wanneer de bitlijn $100\%$ opgeladen is. De bitlijn wordt veronderstelt $100\%$ opgeladen te zijn op het einde van de simulatie en de simulatietijd word voldoende lang gehouden om dit te garenderen. De spanningsval over de cell is belangrijk opdat de cell in van state wisselt gedurende de leescyclus. Aangezien de cell voorgesteld wordt met een weerstand zal dit natuurlijk nooit gebeuren maar dit is wel belangrijk moest er met echte memristors gewerkt worden. De numerieke waarde van de maximale spanningsval over de cel is heel erg afhankelijk van het type memristor. In dit onderzoek wordt er gewerkt met een maximum van 0.5V over de cell \cite{ppt:model}.

\section{vergelijking van verschillende types last}
Voor dit onderzoek worden vier mogelijke kandidaten van last vergeleken: de switchload (figuur \ref{fig:switchload}), de biasload (figuur \ref{fig:biasload}), de diodeload (figuur \ref{fig:diodeload}) en de bulkload (figuur \ref{fig:bulkload})\cite{bulkload}. Eerst wordt er een lineare sweep gedaan op de verschillende lasten (sectie \ref{sec:linload}), waarbij enkel de breetes en bias spanningen worden gesweept. De lengtes van de transistoren worden minimaal gehouden om er voor te zorgen dat de transistoren binnen de pitch van de bitlijn passen. Eens variabiliteit wordt toegevoegd aan de simulatie onder de vorm van monte-carlo (sectie \ref{sec:varload}), zal echter blijken dat er het verschil in bitlijnvoltage te klein is, en zal de lengte van de last transistoren ook moeten worden vergroot (sectie \ref{sec:finaleload}).

\begin{figure}[!ht]
  \centering
  \subfloat[De switch load]{\makebox[.22\textwidth]{ \includegraphics[width=0.12\textwidth] {../fig/hfdst-last-loadtypesswitch.png} \label{fig:switchload}}}
  \subfloat[De bias load]{\makebox[.22\textwidth]{ \includegraphics[width=0.12\textwidth] {../fig/hfdst-last-loadtypesbias.png} \label{fig:biasload}}}
  \subfloat[De diode load]{\makebox[.22\textwidth]{ \includegraphics[width=0.12\textwidth] {../fig/hfdst-last-loadtypesdiode.png} \label{fig:diodeload}}}
  \subfloat[De bulk load]{\makebox[.22\textwidth]{ \includegraphics[width=0.12\textwidth] {../fig/hfdst-last-loadtypesbulk.png} \label{fig:bulkload}}}
  \caption{De verschillende types last}
  \label{fig:loads}
\end{figure}

\subsection{Lineaire sweep op de lasten}\label{sec:linload}
\paragraph{}
De switchload bestaat uit \'{e}\'{e}n pmos transistor die volledig wordt aan of afgesloten. Een lineare sweep met een breedte van de transistor tussen 100nm en 500nm werd gedaan en is geillustreed in figuur \ref{fig:switchloadsim}. Bij het vergroten van de breedte van de transistor zal de weerstand dalen en het verschil tussen de bitlijnen ook. Als we deze last vergelijken men het simpele model uit sectie \ref{sec:simplemodel}, zit de weerstand waarde aan de linker kant van de piek uit figuur \ref{fig:rpiek}. Bij het vergroten van de transistor breedte zal de bitlijn spanning stijgen en de spannings val over de cell dus ook. Verder volgt de risetime ook het simple model uit sectie \ref{sec:simplemodel}, waarbij de risetime daalt bij kleinere weerstandswaardes.

\paragraph{}
De biasload, is een last met twee pmos transistoren in serie. Bovenste van de twee wordt als een switch gebruikt en dus volledig aan of af gesloten. De onderste van de twee wordt op een spanning gebiased. Het voordeel van de biasload is dat men een grotere weerstand kan maken en dus de piek kan bereiken uit figuur \ref{fig:rpiek}. Dit kan men duidelijk zien op de x-assen van figuur \ref{fig:biasloadsim}. Ook hier zijn de breedtes van de transistoren gesweeepts tussen 100nm en 500nm. De bias spanning is tussen 0V en 0.4V gesweept. Een hogere bias spanning brengt echter geen nuttige bij drage. Door dat te kleinste weerstand dat met deze last te maken is ,binnen deze sweeprange, net iets groter is als deze van de switch load, is de biasload ook iets trager. De oplossingen waarbij dit het geval is, hebben echter een onbruikbaar verschill in bitlijn voltages. De spanningsval over de cell is vergeleken met de switch load heel wat hoger maar voor de meeste oplossingen ligt het nog altijd onder de limiet van 0.5V.

\paragraph{}
De diode load bestaat ook uit twee transistoren, de bovenste wordt net als bij de biasload als een switch gebruikt. De onderste is als een diode geconnecteerde transistor gekoppelt. Uit de sweept resultaten (figuur \ref{fig:diodeloadsim}) blijkt dat deze last heel snel is maar veel te kleinen bitlijn voltage verschillen heeft om bruikbaar te zijn.

\paragraph{}
De bulkload werd voorgestelt in de paper van Ren et al. \cite{bulkload} als een goede kandidaat omwille van zijn grote uitgangsimpedantie. Deze last bestaat uit een switch transistor en een bulk geconnecteerde transistor. Deze bulk geconnecteerde transistor wordt op 0V gebiast aangezien deze de beste resultaten gaf. De breedtes van de transistoren zijn gesweept tussen 100nm en 500nm. De resultaten van deze sweep zijn geillustreerd in figuur \ref{fig:bulkloadsim}. In de resultaten kan gezien worden dat deze last zich vergelijkbaar gedraagt als de biaslast. Enkel op het vlak van risetime zijn er oplossingen die beter zijn.

\afterpage{
\begin{figure}
  \centering
  \includegraphics[width=0.67\textwidth]{../fig/hfdst-last-switchload.png}
  \caption{Lineaire sweep van switchload}
  \label{fig:switchloadsim}
 \centering
  \includegraphics[width=0.67\textwidth]{../fig/hfdst-last-biasload.png}
  \caption{Lineaire sweep van biasload}
  \label{fig:biasloadsim}
\end{figure}
}
\afterpage{
\begin{figure}
  \centering
  \includegraphics[width=0.67\textwidth]{../fig/hfdst-last-diodeload.png}
  \caption{Lineaire sweep van diodeload}
  \label{fig:diodeloadsim}
  \centering
  \includegraphics[width=0.67\textwidth]{../fig/hfdst-last-bulkload.png}
  \caption{Lineaire sweep van bulkload}
  \label{fig:bulkloadsim}
\end{figure}
}

\subsection{Het toevoegen van variabiliteit}\label{sec:varload}
Na een selectie te hebben gemaakt van de oplossingen uit de vorige sectie, worden met deze oplossingen nieuwe simulaties gedaan waarbij er variabiliteit is toegevoegt. Deze variabiliteit is toegevoegd op alle transistoren in het test circuit en op de weerstands waarde van de geheugen cellen. Voor de transistoren word er een Pelgrom constante voor vt van $2.5$ gebruikt en voor $\beta$ van $1.2$ gebruikt \cite{ppt:variatie}. Voor de weerstand waarde van de memristors word er een gausische verdeling gebruikt met nominale waardes 7.5k$\Omega$ en 32.5k$\Omega$ en met $\sigma = 0.833k\Omega$. Er worden telkens 500 monte carlo simulaties gedaan per oplossing. Hierna worden de bitlijn voltages van cellen met een HRS en LRS gefit op een gausische distributie. De oplossing met het grootste bitlijn voltage verschil tussen de extrema van HRS en LRS is een biasload met een switch transitor breete van 100nm, een bias transistor breete van 180nm en een bias voltage van 0V. De bitlijn voltage distributie zijn geillustreed op figuur \ref{fig:distbias}. Het voltage verschill tussen de CDF = $0.1\%$ van HRS en CDF = $99.9\%$ van de LRS is 65mV. Dit is niet veel aangezien de distributie van het bitlijn voltage van de referentie cell hier tussen moet passen en er daarna nog marge over moet zijn voor variabiliteit in de senseamplifier. Het aandeel to variabiliteit van beide transistoren in de last is even groot.

\begin{figure}[!ht]
  \centering
  \includegraphics[width=0.67\textwidth]{../fig/hfdst-last-var1.png}
  \caption{Bitlijn voltage distributie voor een biasload}
  \label{fig:distbias}
\end{figure}

Figuur \ref{fig:distref} stelt de distributie van het bitlijn voltage van de referentie cellen voor. Hier bij varieert het aantal referentie cellen van 2 tot 30 en er werd een even groot aantal referentie cell in HRS als LRS gehouden. Zoals gezien kan worden dat met een heel aantal cellen nodig heeft om een distributie breete van 39mV te krijgen. Dit Geeft dan een marge van ongeveer 10mV voor de senseamplifier wat helemaal niet veel is. Daarom word de constraint waarbij de transistor lengte minimaal gehouden word opgeheven in de volgende sectie.

\begin{figure}[!ht]
  \centering
  \includegraphics[width=0.67\textwidth]{../fig/hfdst-last-ref.png}
  \caption{Lineaire sweep van switchload}
  \label{fig:distref}
\end{figure}


\subsection{De transistor lengte vergroten}\label{sec:finaleload}
Om de variabiliteit onder controle tehouden moeten de transistoren vergroot worden. Twee opties worden hiervoor overwogen. De eerste is het toevoegen van een derde transistor in serie. Om de zelfde last impedantie te bekomen als voor 2 transisoren in serie, moeten alle drie transistoren een grote breete hebben wat zou betekenen dat ze groter zijn en minder gevoelig voor mismatch. Een aspect waar niet mee rekening word gehouden in die redenering is de toestand waarin deze transistoren zich bevinden. Bij drie transistoren in serie zal de onderste van de drie zich in near tot sub-theashold bevinden. De stroom in het sub-theshold gebied is exponentieel met de gate-source spanning dit levered grote variatie in de stroom voor kleine vt mismatch. Dit fenemeen zien men niet bij 2 transistoren in serie, aangezien de transistoren hier in het lineare gebied zijn. Daarom word er gekozen voor een tweede optie om de mismatch onder controle te houden namelijk het vergroten van de lengte van de transistor. Als men de lengte vergroot, Stijgt de weerstand wat dan weer gecompenseert kan worden door de breedte ook wat te vergroten. Nu men deze constraint laten varen heeft, word er dan ook geopteert om een switchload ipv een bias load te gebruiken.\\
Figuur \ref{fig:length} geeft de resultaten weer van een sweep van verschillende lengtes en breetes voor een switchload. De resultaten worden voor gestelt in functie van $W/L$ wat een indicatie is voor de weerstand van de transistor. In de bovenste figuur kan men duidelijk een maximun zien voor het verschil in bitlijn voltage zoals in sectie \ref{sec:simplemodel} werd voorspelt. Verder wordt opgemerkt dat er best \'{e}\'{e}n van de oplossingen aan de linkerkant van het maximum gekozen wordt aangezien de spannings val over de cell van de oplossingen aan de rechterkant van het maximum te hoog zijn.
\begin{figure}[!ht]
  \centering
  \includegraphics[width=0.66\textwidth]{../fig/hfdst-last-length.png}
  \caption{Verschillende oplossingen voor de switchload met variabele lengtes en breetes}
  \label{fig:length}
\end{figure}

Voor de finale last wordt er geopteert voor een transistor met lengte gelijk aan 198nm en breedte gelijk aan 300nm. Op figuur \ref{fig:length} word deze aangeduid met de rode lijn. Op figuur \ref{fig:distswitch} word de bitlijn voltage distributie van deze last getoont. Het minimale verschil in bitlijn voltage is bijna 200mV. De distributie van bitlijn voltage van de referentie is ook aangegeven op deze figuur. Deze bestaat hier uit 4 referentie cellen waarvan 2 in HRS en 2 in LRS. Opvallend is dat deze referentie niet in het centrum zit tussen de bitlijn voltages van de cellen. Dit kan opgelost worden door een niet gelijk aantal referentie cellen in HRS en LRS te hebben. Aangezien de standaard deviatie op de bitlijn voltages heel wat beter is nu de lengte van de transistoren ook word gesized, kan er gerust gekozen worden voor een last met een kleiner nominaal verschil in bitlijn voltages \label{anderelast}. Dit kan 2 voordelen met zich mee brengen. Het eerste is dat de spannings val over de cell verlaagt kan worden als met voor een oplossing kiest dan meer links zit van het maximum in figuur \ref{fig:length}. Het Tweede is dat men voor een oplossing kan kiezen waarbij de bitlijn voltages lager zijn wat een energie winst kan opleveren. Ondanks deze voordelen werd er toch geopteert voor de oplossing met het grootste bitlijn voltage verschil.

\begin{figure}[!ht]
  \centering
  \includegraphics[width=0.67\textwidth]{../fig/hfdst-last-var2.png}
  \caption{Bitlijn voltage distributie voor de finale load}
  \label{fig:distswitch}
\end{figure}

\section{Besluit}
\chapter{Sense Amplifier analyse}
\label{sensamp}
Een sense amplifier versterkt kleine signaalverschillen tot rail-tot-rail signalen. Aangezien de uitgangsignalen hiervan ook de uitgelezen bits zijn van het geheugen, is het bovenal belangrijk dat dit op een correcte manier gebeurt, ondanks variabiliteit.
Het is dus logisch om de sense amplifier wat meer te onderzoeken en zodanig te ontwerpen op een robuuste manier, terwijl er ook rekening gehouden wordt met energie en snelheid.

\section{Types SA}
...
\begin{figure}
  \centering
  \includegraphics[scale=0.4]{../fig/hfdstk-sensamp-ourSA.png}
  \caption{een sense amplifier}
  \label{fig:ourSA}
\end{figure}

In wat volgt zal er worden voortgewerkt met de SA van figuur \ref{fig:ourSA}.


\section{Offsetspanning}
Een ideale sense amplifier zal voor elke twee ingangssignalen correct versterken, tenzij de signalen dezelfde zijn, waarna de SA in een metastabiele toestand belandt. In de praktijk is er echter wegens variabiliteit een limiet voor het ingangsspanningsverschil waarbij er correct versterkt wordt. Deze limiet heet de offsetspanning en wordt geïllustreerd in figuur \ref{fig:offset}. De offsetspanning van een SA is in de ontwerpfase een stochastische variabele met gemiddelde 0V, pas nadat een chip gefabriceerd is ligt de offsetspanning definitief vast [al kan het zijn dat deze met de tijd nog verandert].
Er zijn 2 manieren waarop men de offsetspanning van een systeem kan aanpakken: ofwel ontwerp je het systeem zodanig dat het verschil van de ingangssignalen van de SA groot genoeg is zodat ze [in 99,9\% van de gevallen] niet groter is dan de offsetspanning, ofwel bouw je een mechanisme in waarbij je na fabricatie de offsetspanning meet en vervolgens compenseert. In dit werk is gekozen voor het eerste.
Hiervoor is het wel belangrijk te onderzoeken wat de verdeling is van de offsetspanning, dit wordt gedaan in de volgende sectie.

\begin{figure}
  \centering
  \includegraphics[scale=0.4]{../fig/hfdstk-sensamp-offset.png}
  \caption{Illustratie van offsetspanning}
  \label{fig:offset}
\end{figure}

\section{Sensitiviteitsanalyse}
De SA wordt gerealiseerd als een circuit met transistors. Elke transistor heeft 2 stochastische parameters met een normale verdeling, nl. $\Delta V\textsubscript{t}$ en $\Delta \beta$. De spreiding van deze verdelingen is gekend: $\sigma_{\Delta V_{t}} = \frac{A\textsubscript{V\textsubscript{t}}}{\sqrt{W L}}$ en $\sigma_\frac{{\Delta \beta}}{\beta} = \frac{A_{\beta}}{\sqrt{W L}}$. Met een sensitiveitsanalyse kan men uit deze standaardafwijkingen de standaardafwijking van de offsetspanning $\sigma_{V_{offset}}$ berekenen. Hierbij wordt verondersteld dat de stochastische variabele $V_{offset}$ een lineaire combinatie is van de normaal verdeelde afwijkingen $(\Delta V_{t})_{i}$ en $(\frac{\Delta_{\beta}}{\beta})_{i}$: $V_{offset}=\sum\limits_{i=1}^{N} a_{i} (\Delta V_{t})_{i} + b_{i} (\frac{\Delta_{\beta}}{\beta})_{i}$.
$a_{i}$ en $b_{i}$ zijn de gevoeligheden van de offset naar de variatieparameters: $a_{i}=\frac{\partial V_{offset}}{\partial (\Delta V_{t})_{i}}$ en $b_{i}=\frac{\partial V_{offset}}{\partial (\frac{\Delta_{\beta}}{\beta})_{i}}$.
Voor een dergelijke variabele geldt dan: $\sigma_{V_{offset}}=\sqrt{\sum\limits_{i=1}^{N} a_{i}^{2} (\sigma_{\Delta V_{t}})_{i}^{2} + b_{i}^{2} (\sigma_{\frac{\Delta_{\beta}}{\beta}})_{i}^{2}}$.

Er moet wel geverifieerd worden of de stelling dat er een lineaire afhankelijkheid is tussen $V_{offset}$ en de variatieparameters gegrond is.
Dit kan gedaan worden aan de hand van een analyse waarbij elke variatieparameter afzonderlijk gesweept wordt. 

\subsection{Sensitiviteitsanalyse op een minimale SA}
In figuur \ref{fig:min-sensanalysis} wordt het resultaat getoond voor een dergelijke analyse bij een SA met minimale afmetingen, merk op dat de richtingscoëfficient van deze curves gelijk is aan $a_{i} (\Delta V_{t})_{i}$ en $b_{i} (\frac{\Delta_{\beta}}{\beta})_{i}$.  In tabel \ref{tab:min-sensanalysis} worden de resultaten en de resulterende standaardvariatie van de SA getoond.
Er moet opgemerkt worden dat er bij deze simulatie slechts gesweept werd voor de variatieparameters van -4$\sigma$ tot 4 $\sigma$. Dit is om de reden dat voor de minimale transistoren de standaardvariatie het grootst is. In de Spectre-simulaties zouden transistoren voor te grote negatieve $\beta$-mismatch stroom leveren in de omgekeerde richting. Deze situatie zal fysisch nooit optreden.

\begin{table}
\begin{tabular}{cccccc}
\hline 
Transistor & Parameter & Richtingscoëfficiënt [$\frac{mV}{\sigma}$] & W [nm] & L [nm] & $\sigma$ \\ 
\hline 
Mupbar & Vt & 22.733 & 100 & 45 & 37.2678mV \\ 
Mup & Vt & -22.250 & 100 & 45 & 37.2678mV \\ 
Mupbar & $\beta$ & 13.583 & 100 & 45 & 17.8885\% \\ 
Mpassn & $\beta$ & 13.467 & 100 & 45 & 29.8142\% \\ 
Mpassbarn & $\beta$ & -13.117 & 100 & 45 & 29.8142\% \\ 
Mup & $\beta$ & -13.033 & 100 & 45 & 17.8885\% \\ 
Mdownbar & $\beta$ & -9.383 & 100 & 45 & 29.8142\% \\ 
Mdown & Vt & -9.267 & 100 & 45 & 42.0381mV \\ 
Mdownbar & Vt & 9.233 & 100 & 45 & 42.0381mV \\ 
Mdown & $\beta$ & 8.217 & 100 & 45 & 29.8142\% \\ 
Mpassp & $\beta$ & -4.50 & 100 & 45 & 17.8885\% \\ 
Mpassbarp & $\beta$ & 4.383 & 100 & 45 & 17.8885\% \\ 
Mpassbarp & Vt & 0.70 & 100 & 45 & 37.2678mV \\ 
Mpassp & Vt & -0.70 & 100 & 45 & 37.2678mV \\ 
Mbottom & $\beta$ & 0.083 & 100 & 45 & 29.8142\% \\ 
Mbottom & Vt & -0.033 & 100 & 45 & 42.0381mV \\ 
Mpassbarn & Vt & 0 & 100 & 45 & 42.0381mV \\
Mpassn & Vt & 0 & 100 & 45 & 42.0381mV \\
Mtop & Vt & 0 & 100 & 45 & 37.2678mV \\
Mtop & $\beta$ & 0 & 100 & 45 & 17.8885\% \\
\hline 
\hline & $\sigma_{Voffset}$: & 45.6813mV & & & \\
\hline
\end{tabular} 
\caption{Sensitiviteitsanalyse van de minimale SA}
\label{tab:min-sensanalysis}
\end{table}

Opmerkelijk bij deze analyse is dat er een significante bijdrage is van de pass-gates door $\beta$-mismatch. Een nadere observatie leert dat deze bijdrage optreedt door ladingsinjectie van de pass-gates die niet meer gematched is (zie figuur \ref{fig:chargeinjectionmismatch}).
Hierbij moet wel worden opgemerkt dat voor deze simulatie er geen overlap is tussen het controlesignaal op de passgate aan te zetten en het signaal om de SA te activeren.
De reden hierachter is dat als er overlap tussen deze signalen is, de SA ook de BL zou trachten op te laden. Hierbij zou er moeten ingeboet worden aan snelheid en het zou ook extra energie kosten.

Men kan argumenteren dat er een korte overlap zou kunnen toegelaten zijn, waarna er voldoende spanningsverschil tussen de 2 ingangs-uitgangsknopen zou opgebouwd zijn opdat de ladingsinjectie geen effect meer kan hebben op het eindresultaat. Een tegenargument is dat de timing hiervoor te precies moet zijn.
\begin{figure}
  \centering
  \includegraphics[scale=0.4]{../fig/hfdstk-sensamp-chargeinjectionmismatch.png}
  \caption{Door $\beta$-mismatch is ladingsinjectie van de pass-gates niet meer gematched en gaat de SA foutief latchen}
  \label{fig:chargeinjectionmismatch}
\end{figure}

\subsection{RC-latch-effect}
\label{RC-latch-effect}
De situatie waarbij er volledige overlap is tussen de controle signalen kan vereenvoudigd worden opgesteld met de situatie van figuur \ref{fig:RC-latch}. De pass-gate die aanstaat wordt voorgesteld als een weerstand, de pass-gate in het local block en diens parasitaire capaciteit wordt verwaarloosd. CL bedraagt voor deze simulatie 46 fF, het equivalent voor een BL waaraan 256 cellen hangen. Cint bedraagt voor een SA met minimale transistorafmetingen 161 aF. Wanneer het dynamisch latch-gedrag bekeken wordt voor verschillende waardes van R, treedt er een merkwaardig effect op (zie figuur \ref{fig:RC-latch-sim}): voor voldoende grote waardes van R lijkt het alsof de grote capaciteit ontkoppeld is van de latch tot op een zeker tijdstip, waarna een veel tragere settling optreedt.
\begin{figure}
  \centering
  \includegraphics[scale=0.4]{../fig/hfdstk-sensamp-RC-latch.png}
  \caption{Simulatieopstelling voor het RC-latch-effect}
  \label{fig:RC-latch}
\end{figure}
\begin{figure}
  \centering
  \includegraphics[scale=0.4]{../fig/hfdstk-sensamp-RC-latch-sim.png}
  \caption{Simulatieresultaten voor het RC-latch-effect:de 2 ingangs-uitgangsknopen zijn voorgeladen op 400mV en 380mV. Na 1,6ns wordt de SA aangezet. De SA is ideaal voor deze simulatie.}
  \label{fig:RC-latch-sim}
\end{figure}
De verklaring ligt in het feit dat CL zich voor hoge frequenties als een kortsluiting gedraagt (zie figuur \ref{fig:RC-latch-explain}), een plotse stroom vloeit door de weerstand en hierdoor onstaat er een spanningsval over de weerstand. Hierna gaat er op veel lagere frequenties een spanning beginnen op te bouwen over de capaciteit waardoor de ingangs-uitgansknopen volledig kunnen laden/ontladen tot VDD en VSS.
\begin{figure}
  \centering
  \includegraphics[width=\textwidth]{../fig/hfdstk-sensamp-RC-latch-explain.png}
  \caption{Vergelijking situatie met voorgeladen (eindige) capaciteit en situatie met spanningsbron (oneindige capaciteit)}
  \label{fig:RC-latch-explain}
\end{figure}
Gevolgen van wanneer dit effect optreedt is dus dat het nuttige signaal zich snel - alsof er helemaal geen last aanhangt - en lineair opbouwt en dat er geen AC-signaal is over de condensator. Een analyse van de respons van een RC-circuit op een lineair stijgende spanningsbron geeft meer duidelijkheid voor de voorwaarden waarop het RC-latch-effect optreedt (zie figuur \ref{fig:RC-latch-maplecircuit}). De respons van de spanning over de capaciteit is $Vcap(t) =  at - aRC(1-e^{-{\frac {t}{CR}}})$. Uit deze uitdrukking blijkt dat het RC-latch-effect optreedt wanneer de latch zonder last snel is (a<<1) en/of wanneer het RC-product hoog is (RC>>1).
Wanneer het effect zich voordoet zijn latching en RC-respons onafhankelijke processen. Wanneer de voorwaarden niet meer zo uitgesproken zijn, gaan deze processen met elkaar interfereren en is het moeilijk dit gecombineerde proces wiskundig te beschrijven.
\begin{figure}
  \centering
  \includegraphics[scale=0.4]{../fig/hfdstk-sensamp-RC-latch-maplecircuit.png}
  \caption{Circuit voor analyse voorwaarden RC-latch-effect}
  \label{fig:RC-latch-maplecircuit}
\end{figure}

Conclusie van het RC-latch effect is dat de timing helemaal niet zo kritisch is: in theorie hoeft de overlap slechts even lang te duren als de delay van de SA wanneer er geen last op is aangesloten, maar het is niet erg als de overlap wat langer duurt.
De pass-gates mogen ook minimaal zijn, om hun aanweerstand te vergroten zodat het effect kan optreden.
In geval verder zou gewerkt worden met een SA zonder overlap met pass-gate-enable en SA-enable, zouden de pass-gates moeten geschaald worden om de mismatch te minimaliseren. Dit zou wel betekenen dat er per schakeling van de passgates een grotere hoeveelheid lading wordt geïnjecteerd.


\subsection{Sensitiviteitsanalyse voor minimale SA - vervolg}

In tabel \ref{tab:min-sensanalysis-overlap} worden de resultaten van een nieuwe sensitiveitsanalyse getoond voor een minimale SA, ditmaal waarbij er dus overlap is tussen pass-gate-enable en SA-enable. De spreiding van de offsetspanning is wel degelijk gedaald. Toch heeft de mismatch van de passgates nog steeds een significante bijdrage, dit kan verklaard worden a.d.h. van figuur \ref{fig:RC-latch}: de 2 weerstanden zijn niet gematcht, deze mismatch treedt op door zowel $\beta$- als $V_{T}$-mismatch van de passgates. De bijdrage van de mismatch van de differentiële paren daalt echter door de interactie met de passgates.

\begin{table}
\begin{tabular}{cccccc}
\hline 
Transistor & Parameter & Richtingscoëfficiënt [$\frac{mV}{\sigma}$] & W [nm] & L [nm] & $\sigma$ \\ 
\hline 
Mupbar & Vt & 15.250 & 100 & 45 & 37.2678mV \\ 
Mup & Vt & -14.900 & 100 & 45 & 37.2678mV \\ 
Mdownbar & Vt & 10.317 & 100 & 45 & 42.0381mV \\ 
Mdown & Vt & -9.916 & 100 & 45 & 42.0381mV \\
Mpassn & $\beta$ & 8.433 & 100 & 45 & 29.8142\% \\ 
Mpassbarn & Vt & 6.833 & 100 & 45 & 42.0381mV \\
Mpassbarn & $\beta$ & -6.767 & 100 & 45 & 29.8142\% \\
Mpassn & Vt & -6.733 & 100 & 45 & 42.0381mV \\ 
Mupbar & $\beta$ & 6.383 & 100 & 45 & 17.8885\% \\ 
Mup & $\beta$ & -6.133 & 100 & 45 & 17.8885\% \\ 
Mdownbar & $\beta$ & -5.633 & 100 & 45 & 29.8142\% \\  
Mdown & $\beta$ & 5.233 & 100 & 45 & 29.8142\% \\ 
Mpassp & Vt & 0.233 & 100 & 45 & 37.2678mV \\
Mpassbarp & Vt & -0.233 & 100 & 45 & 37.2678mV \\
Mtop & Vt & 0.167 & 100 & 45 & 37.2678mV \\  
Mpassp & $\beta$ & -0.167 & 100 & 45 & 17.8885\% \\ 
Mpassbarp & $\beta$ & 0.167 & 100 & 45 & 17.8885\% \\ 
Mtop & $\beta$ & 0.117 & 100 & 45 & 17.8885\% \\
Mbottom & $\beta$ & 0.050 & 100 & 45 & 29.8142\% \\ 
Mbottom & Vt & 0.0333 & 100 & 45 & 42.0381mV \\ 
\hline 
\hline & $\sigma_{Voffset}$: & 31.7172mV & & & \\
\hline
\end{tabular} 
\caption{Sensitiviteitsanalyse van de minimale SA met overlap tussen passenable en latchenable}
\label{tab:min-sensanalysis-overlap}
\end{table}

\subsection{Sensitiviteitsanalyse voor gebruikte SA}

In tabel \ref{tab:ourSA-sensanalysis-overlap} worden de resultaten van een sensitiviteitsanalyse getoond voor de SA die gebruikt wordt in het finale geheugenontwerp. Deze is gekozen aan de hand van de resultaten van de paretosimulatie in de volgende sectie.


\begin{table}
\begin{tabular}{cccccc}
\hline 
Transistor & Parameter & Richtingscoëfficiënt [$\frac{mV}{\sigma}$] & W [nm] & L [nm] & $\sigma$ \\ 
\hline 
Mup & Vt & -4.283 & 1700 & 45 & 9.0388mV \\ 
Mupbar & Vt & 4.283 & 1700 & 45 & 9.0388mV \\ 
Mpassn & Vt & -3.767 & 500 & 45 & 18.8000mV \\
Mpassbarn & Vt & 3.650 & 500 & 45 & 18.8000mV \\
Mpassbarn & $\beta$ & -3.000 & 500 & 45 & 13.3333\% \\ 
Mpassn & $\beta$ & 3.000 & 500 & 45 & 13.3333\% \\ 
Mup & $\beta$ & -1.767 & 1700 & 45 & 4.3386\% \\ 
Mupbar & $\beta$ & 1.767 & 1700 & 45 & 4.3386\% \\ 
Mdown & Vt & -1.100 & 1500 & 45 & 10.8542mV \\ 
Mdownbar & Vt & 1.100 & 1500 & 45 & 10.8542mV \\
Mdown & $\beta$ & 0.8333 & 1500 & 45 & 7.6980\% \\
Mdownbar & $\beta$ & -0.8333 & 1500 & 45 & 7.6980\% \\  
Mpassp & Vt & 0.1667 & 500 & 45 & 16.6667mV \\ 
Mpassp & $\beta$ & 0.1667 & 500 & 45 & 8\% \\ 
Mpassbarp & Vt & -0.1667 & 500 & 45 & 16.6667mV \\
Mpassbarp & $\beta$ & -0.1667 & 500 & 45 & 8\% \\ 
Mtop & $\beta$ & 0.1333 & 900 & 45 & 5.9628\% \\ 
Mtop & Vt & 0.1000 & 900 & 45 & 12.4226mV \\ 
Mbottom & $\beta$ & -0.06667 & 500 & 45 & 13.3333\% \\ 
Mbottom & Vt & 0.03333 & 500 & 45 & 18.8000mV \\ 
\hline 
\hline & $\sigma_{Voffset}$: & 9.6125mV & & & \\
\hline
\end{tabular} 
\caption{Sensitiviteitsanalyse van de SA in het finale geheugen, er is overlap tussen de controlesignalen}
\label{tab:ourSA-sensanalysis-overlap}
\end{table}

\section{Paretosimulatie}
In het beginstadium van het ontwerp is nog niet duidelijk wat de impedantie aan de BL wordt. Het is deze impedantie die bepaalt wat het spanningsverschil is tussen het datasignaal en het referentiesignaal aan de sense amplifier. Bovendien kan het zijn dat er midden in het ontwerp besloten wordt om een andere impedantie te kiezen om alsnog te optimaliseren naar een andere variabele.
Natuurlijk is het mogelijk om één SA te gebruiken die voor elke impedantie een correcte en snelle werking zou garanderen. Dit zou echter een verspilling zijn van energie. In deze sectie wordt een pareto-oppervlak opgesteld waarbij er voor elk spanningsverschil de snelste en energiezuinigste SA-ontwerpen worden gekozen.

\subsection{Opstelling}
Uit een verzameling van allerhande SA [dit zijn sense amplifiers waarvan de transistoren verschillend geschaald zijn - differentiële paren hebben zelfde afmetingen] worden enkel de pareto-optimale SA uitgekozen. De pareto-criteria zijn $\Delta V$, snelheid en dynamische energie. 

Voor deze opstelling worden de pass-gates weggelaten van de SA [dit is geoorloofd zoals bleek uit de sensitiviteitsanalyse], de last aan de ingangs-uitgangsknopen is een simpele CMOS inverter. De knopen zijn voorgeladen op 2 spanningen: 0,4V en 0,4V - $\Delta V$. Na 0,5ns wordt de SA aangezet en wordt de tijd gemeten tot wanneer de ingangs-uitgangsknopen geladen of ontladen zijn tot 99,9\% van hun finale waarde (VDD of VSS). Dit is wellicht een te strenge methode om de snelheid van de SA te bepalen aangezien de inverters al eerder zullen schakelen. Indien de snelheid van de 2 knopen verschilt, zal de traagste tijd genomen worden. De dynamische energie wordt opgemeten van het moment dat de SA wordt aangeschakeld tot dit tijdstip. Ook het statisch vermogen van de SA wordt opgemeten wanneer de ingangs-uitgangsknopen VDD en VSS bereikten. Uiteraard wordt ook geverifieerd ofdat de SA wel correct heeft gelatcht.

Per sense amplifier worden er 250 Monte Carlo simulaties gedaan met deze opstelling. Indien de SA niet elke keer correct functioneerde, wordt de SA verworpen. Latchte de SA wel elke keer correct, wordt het gemiddelde van de delay, dynamische energie  en statisch vermogen opgeslagen. 

\subsection{Resultaten}
Op figuur \ref{fig:pareto} zijn de pareto-optimale resultaten getoond van de groep sense amplifiers. Het doel van deze simulatie is veeleer om de transistorafmetingen te situeren in functie van deze optimalizatievariabelen. Voor deze simulatieopstelling kan men enkel zeggen dat de kans dat de offsetspanning lager is dan $\Delta V$ minstens $1 -\frac{1}{250}$ is. Dit is een veel te kleine garantie voor een sense amplifier die misschien wel miljoenen keren zal gefabriceerd worden. Voor meer informatie over de verdeling van de offsetspanning te krijgen moet de standaardafwijking berekend worden met de sensitiviteitsanalyse.

\begin{figure}
  \centering
  \includegraphics[width=0.8\textwidth]{../fig/hfdstk-sensamp-pareto2.png}
  \caption{De pareto-optimale sense amplifiers}
  \label{fig:pareto}
\end{figure}

\section{Besluit}
In dit hoofdstuk werd dieper ingegaan op de sense amplifiers, die het kleine spanningsverschil tussen datasignaal en referentiesignaal correct moet versterken tot VDD en VSS. De belangrijkste eigenschap van de SA is de offsetspanning door transistorvariaties. Deze kan voldoende klein gemaakt worden door de transistoren voldoende op te schalen. De offsetspanning kan statistisch beschreven worden met behulp van een sensitiviteitsanalyse. Tenslotte worden er ook uit een grote groep SA de pareto-optimale gekozen. De resultaten geven een idee van de grootteordes van transistorafmetingen voor een bepaalde offetspanning, snelheid en dynamische energie.

\chapter{Omringende logica}
\label{periphery}

Een heleboel logische blokken, zoals decoders, drivers, pass-gates en buffers zitten verwerkt in de geheugenstructuur.
In dit hoofdstuk worden deze componenten van wat dichterbij onderzocht.

\section{Decoders}

\subsection{Vergelijkende studie}

\section{Buffers}

\section{BL- en WL-drivers}

\section{Passgates}


\section{Besluit}


\chapter{Timing en optimalizatie}
\label{timing-optimization}
Voor een correcte werking van het geheugen, is het van belang dat de verschillende controlesignalen in een welbepaalde volgorde verwerkt en doorgegeven worden.
Door al de signalen even snel te maken als het kritisch pad is er ruimte voor optimalisatie. In het eerste deel van dit hoofdstuk zal de invloed van architectuur en sizing onderzocht worden op de timing van de signalen. De constraints en vrijheidsgraden die hieruit volgen zullen dan gebruikt worden in het tweede deel van dit hoofdstuk om een optimale architectuur te bepalen.

\section{Timing}
\label{timing}
Het ontwerp van het geheugen in dit werk gaat tot het niveau van het global block (GB) \ref{globalblock}, hierbij wordt de veronderstelling gemaakt dat alle signalen tegelijkertijd binnen komen in het GB. Hierna propageren de signalen door logische poorten tot ze verschillende transistoren rond de BL aansturen. De aansturing van deze transistoren leidt tot de eerste kritische timing constraints. Vervolgens worden de passgates en SA aangesloten, deze zullen de tweede timing constraints bevatten.

\subsection{Kritische timing voor het (de)selecteren cell}
\label{optimization}
\paragraph{}
Timingproblemen die het gevolg zijn van het (de)selecteren van de cel ontstaan door een verschil in timing voor (de)selecteren van de lastimpedantie en cel. Indien de load geactiveerd wordt voor de cel geactiveerd is, zal de bitline vroegtijdig beginnen opladen naar de voedingsspanning. Wanneer de cel dan geselecteerd is zal de bitline naar een spanning getrokken worden die overeen komt met de referentiespanning of een spanning van een cel in RHS of LRS. Afhankelijk van het tijdsverschill tussen deze twee gebeurtenissen, zal de bitline terug omlaag getrokken worden, wat resulteert in een energieverspilling. Dit wordt geïllustreerd in figuur \ref{fig:critisch_timing1}. Indien de cel gedeselecteerd wordt voordat de load gedeselecteerd is, zal de bitline ook opladen naar de voedingsspanning. Dit heeft als gevolg dat het ontladen van de bitlijn langer zal duren of het risico op een leesverstoring vergroot zal worden. Het overbodig opladen resulteert eveneens in een energieverspilling. De passgate die aan de BL hangt, wordt aangestuurd door de controlelogica. Door de opbouw van deze logica, heeft de uitgangsknoop achter de passgate één inverterdelay tijd om te ontladen. Vaak is dit niet voldoende, en zal de uitgangsknoop van het LB niet volledig ontladen zijn. Dit wordt geïllustreerd in figuur \ref{fig:critisch_timing2}.   Dit heeft geen nadelige gevolgen omdat de capaciteit op dit knooppunt heel klein is en er bijgevolg een verwaarloosbare ladingsherverdeling is in de volgende leescyclus. 


\begin{figure}[ht]
\centering
\subfloat[Overshoot van BL omdat load geactiveerd wordt\newline voor WL geactiveerd is]{ \includegraphics[width=0.55\textwidth] {../fig/hfdstk-timing-crit1.png} \label{fig:critisch_timing1}}
\subfloat[Overshoot van BL omdat load degeactiveerd wordt nadat WL degeactiveerd is]{ \includegraphics[width=0.40\textwidth] {../fig/hfdstk-timing-crit2.png} \label{fig:critisch_timing2}}
\caption[Timingproblemen bij de bitline]{Timingproblemen bij de bitline}
\clearpage
\end{figure}

\paragraph{}
Er wordt verondersteld dat alle adressignalen tegelijkertijd in het GB aankomen. De timinganalyse start dan ook vanaf dit tijdstip. Het circuit- en timingsdiagram van de logica in het GB worden geïllustreerd in figuren \ref{fig:gb_timing1} en \ref{fig:gb_timing2}. T1 en T2 stellen de momenten voor dat de signalen uit de BL- en WL-decoder komen. T3 stelt het moment voor dat het signaal uit de referentiebuffer komt. T1 en T3 zouden op het zelfde moment moeten aankomen om een optimale timing te hebben. Indien dit niet het geval is, zullen de referentie-BL al geactiveerd zijn voordat de data-BL opgeladen wordt. Indien er een groot aantal referentiebitlijnen zijn, zal dit resulteren in een grote energieverspilling. Er zijn twee opties om de correcte volgorde tussen T1,T2 en T3 te garanderen. De eerste is het kiezen van een kleine BL-decoder en een grote WL-decoder. Dit zal voor een kleine T1 zorgen door een kleine delay in de bitlijndecoder. Dit zal tevens een grotere T3 geven omdat de referentiebuffer een grotere last heeft om op te laden. Een evenwicht kan zo gevonden worden om T1 en T3 op het zelfde moment te doen verschijnen. Deze eerste optie beperkt de mogelijke architectuur aanzienlijk en zal het verwezenlijken van andere timingconstraints voor T2 onmogelijk maken, zoals later zal blijken.\\ 
De tweede optie voor het matchen van T1 en T3 is het uitstellen van T3. Dit uitstel kan gerealiseerd worden door het invoeren van delayelementen of door de buffer suboptimaal te ontwerpen. Het invoeren van delay(vertragings)elementen zorgt in de praktijk echter meestal voor een te grote bijkomende delay. Daarom werd er in het finale ontwerp een buffer ontworpen die niet optimaal is wat snelheid betreft. Om het energieverbruik van de referentiebitlijnen verder in te perken werden niet al de bitlijnen in de array gebruikt voor het genereren van het referentiesignaal.

\begin{figure}[!ht]
  \centering
  \includegraphics[scale=0.6]{../fig/hfdstk-timing-gb1.png}
  \caption[Global block:logica]{Global block logica}
  \label{fig:gb_timing1}
\end{figure}

\begin{figure}[!ht]
  \centering
  \includegraphics[scale=0.9]{../fig/hfdstk-timing-gb2.png}
  \caption[Global block:timing]{Timing global block}
  \label{fig:gb_timing2}
\end{figure}

\paragraph{}
Eens de sigalen uit de decoders komen worden deze gestuurd naar de controlelogica voor de memory array. Het circuit- en timingdiagram hiervan staan geïllustreerd in figuren \ref{fig:lbcell_timing1} en \ref{fig:lbcell_timing2}. De cel zou vroeger dan of gelijk met de last moeten aangeschakeld worden. Op het timingdiagram wordt dit geïllustreerd als T4 = T5 = T6. Door de implementatie van de logica is dit niet mogelijk aangezien er altijd een inverterdelay verschil is tussen T4 en T5. Deze vertraging is minimaal en kan getolereerd worden omdat de bitlijn in elk geval moet opgeladen worden tot minimum $V_{LRS}$. Bij lage voedingsspanningen wordt dit probleem echter meer uitgesproken. T6 wordt bepaald door WL-decoder en -buffers. Deze vertraging zou zodanig ontworpen moeten worden dat deze vroeger dan of gelijk met T5 valt. Bij het afschakelijken van de cel zijn de omgekeerde voorwaarden nodig. De last zou namelijk vroeger dan of gelijk met de cel moeten afgeschakeld worden. Deze voorwaarde is voldaan als T7 voor T8 en T9 komt. Door de inverter is T7 altijd voor T8. T9 daarentegen wordt bepaald door de WL-decoder en -buffer en zou voor T7 moeten komen. \\
Zoals uit figuur \ref{fig:lbcell_timing1} blijkt, zijn de meeste controlesignalen om een datacel uit te lezen niet onafhankelijk, enkel de timing van het WL-signaal kan vrij aangepast worden. Deze moet geselecteerd worden vooraleer de sourceline geselecteerd is en mag pas gedeselecteerd worden nadat de last gedeselecteerd is. \\
De timing van de wordline wordt expliciet bepaald door de grootte van de WL-decoder en impliciet door de grootte van de BL-decoder. De BL-decoder bepaalt namelijk de delay van de WL-buffer. Figure \ref{fig:decoder_dep} geeft de delay van verschillende groottes van WL-decoders en -buffers i.f.v. verschillende groottes van BL-decoders weer.

\begin{figure}[!ht]
  \centering
  \includegraphics[scale=0.6]{../fig/hfdstk-timing-lbcell1.png}
  \caption[Data-array:logica]{Controlelogica data-array}
  \label{fig:lbcell_timing1}
\end{figure}

\begin{figure}[!ht]
  \centering
  \includegraphics[scale=0.9]{../fig/hfdstk-timing-lbcell2.png}
  \caption[Data-array:timing]{Timing controlelogica data-array}
  \label{fig:lbcell_timing2}
\end{figure}

\begin{figure}[!ht]
  \centering
  \includegraphics[scale=0.6]{../fig/hfdstk-timing-decoder-dep.png}
  \caption[Delay van WL-decoders en -buffers i.f.v. BL-decoders]{Delay van WL-decoders en -buffers i.f.v. aantal bitlijnen}
  \label{fig:decoder_dep}
\end{figure}

\paragraph{}
De timingvoorwaarden voor het selecteren en deselecteren van de referentiecellen zijn dezelfde als die van de datacellen. Het circuit- en timingdiagramma hiervan zijn geïllustreerd in figuren \ref{fig:lbref_timing1} en \ref{fig:lbref_timing2}. Anders als bij de datacellen is er aan de timingvoorwaarden al automatisch voldaan met deze logica. Dit omdat de WLs worden aangestuurd door een signaal dat rechtstreeks van de BL-decoder komt. Dit signaal word dan vertraagd met twee invertoren om de juiste timing te verwezenlijken.


\begin{figure}[!ht]
  \centering
  \includegraphics[scale=0.6]{../fig/hfdstk-timing-lbref1.png}
  \caption[Referentie-array:logica]{Controlelogica referentie-array}
  \label{fig:lbref_timing1}
\end{figure}

\begin{figure}[!ht]
  \centering
  \includegraphics[scale=0.9]{../fig/hfdstk-timing-lbref2.png}
  \caption[Referentie-array:timing]{Timing controlelogica referentie-array}
  \label{fig:lbref_timing2}
\end{figure}

\subsection{Kritische timing voor het uitlezen van de cel}
Eens de cel geselecteerd is wordt de bitline opgeladen. De volgende stap is dit signaal te voeden aan de sense amplifier. Het signaal wordt eerst door een eerste passgate geleid om uit het local block te geraken. Vervolgens wordt het signaal door een tweede passgate geleid die als sample-and-hold dient voor de sense amplifier. Figuren \ref{fig:sa_timing1} en \ref{fig:sa_timing2} illustreren het circuit en timing rond de sense amplifier. Eens de BL wordt aangesproken wordt de eerste passgate automatisch geactiveerd zoals uitgelegd in de vorige paragrafen. T19 stelt het tijdstip voor wanneer de tweede passgate aangezet moet worden. Deze timing is niet cruciaal: de tweede passgate mag zowel voor als na de eerste passgate geactiveerd worden. Het tijdstip waarop deze passgate wordt afgeschakeld (T22) is daarentegen wel belangrijk. Dit moet namelijk gebeuren voordat de eerste passgate afgesloten is (T21), indien niet zullen er 2 ladingsinjectie optreden i.p.v. één. Om een zo snel mogelijke latching van de sense amplifier te verkrijgen, is het tijdstip waarop de sense amplifier (T20) geactiveerd wordt belangrijk. Wanneer de sense amplifier juist wordt aangesloten treedt er het RC-latch-effect op waarbij de SA zich gedraagt alsof er geen last aan hangt. Dit effect werd beschreven in sectie \ref{RC-latch-effect}. Na deze snelle fase, gaan de ingangs-uitgangsknopen van de SA veel trager opladen en gaat de SA de BL ook op- of ontladen. Om een snelle latching te verkrijgen moet de tweede passgate dus zo snel mogelijk na de snelle fase afgeschakeld worden. Eens de ingangs-uitgangsknopen van de SA gelatcht zijn, mag de SA gedesactiveerd worden.

\begin{figure}[!ht]
  \centering
  \includegraphics[scale=0.6]{../fig/hfdstk-timing-sa1.png}
  \caption[SA:logica]{Logica rond SA}
  \label{fig:sa_timing1}
\end{figure}

\begin{figure}[!ht]
  \centering
  \includegraphics[scale=0.9]{../fig/hfdstk-timing-sa2.png}
  \caption[SA:timing]{Timing logica rond SA}
  \label{fig:sa_timing2}
\end{figure}

\section{Analyse verschillende geheugenconfiguraties}
\paragraph{}
Het finale geheugen is 1Mbit groot. Heel wat configuraties zijn mogelijk om dit te verwezenlijken. Om deze mogelijkheden wat in te perken wordt volgende beperking opgelegd: het aantal WLs moet groter dan of gelijk zijn aan het aantal BLs. Bij deze configuraties zal het ontladen van de bitlijn sneller verlopen dan bij configuraties met meer BLs dan WLs. Dit levert 20 mogelijke configuraties voor NoWLpB, NoBLpLB en NoGB. Deze configuraties worden vergeleken op basis van oppervlakte, energieverbruik en leessnelheid. 

\subsection{Evaluatie criteria voor de geheugenconfiguraties}
De oppervlakte wordt berekend op basis van de lengtes en breedtes van het totaal aantal transistoren (behalve de celtransistoren). Verbindingslijnen worden niet meegerekend in de berekeningen van de oppervlakte van de logica. Voor de lengte van de geheugencellen wordt 1.5*6F genomen en voor de breedte wordt 2*6F genomen \cite{ppt:cosemans}. Hoewel deze afmetingen voor een MTJ geheugen cell zijn, geven ze een goede schatting van de oppervlakte van een 1T1R-cel. Deze oppervlakte omvat ook de oppervlakte van bitlijn, woordlijn en sourcelijn meegerekend. \\
Het energie verbruik wordt berekend door de stroom van de voedingsspanning te integreren over de tijd en te vermenigvuldigen met de voedingsspanning. De signalen die binnen komen in een global block zijn in SPICE simulaties afkomstig van ideale spanningsbronnen. Dit heeft als gevolg dat er een ladingsinjectie optreedt naar de voedingsspanningsbron (zie bijlage \ref{app:chargeinj}). Dit heeft een beperkte invloed op de energieberekeningen. Deze invloed wordt evenwel niet meegenomen in de analyse. \\
De leessnelheid is afhankelijk van de verschillende controlesignalen in de leescyclus. De simulatie-opstelling is als volgt: de leescyclus begint wanneer de signalen binnen komen in het global block. De SA wordt aangezet wanneer het verschil tussen de data- en referentiesignaal 100mV bedraagt. Omdat er niet gewacht wordt op het tijdstip wanneer de BL volledig opgeladen is, wordt het verschil in leessnelheid tussen geheugens met een klein aantal woordlijnen en geheugens met een groot aantal woordlijnen vergroot. Geheugens met iets meer woordlijnen worden zo in de race gehouden. In het finale geheugenontwerp zal de leessnelheid verder opgedreven worden door dit 100mV spanningsverschil te verkleinen. Verder wordt er ook altijd een HRS-cel uitgelezen aangezien deze de bitlijn langer moet opladen om tot aan de 100mV verschildrempel te komen wat een realistischere leessnelheid geeft. De leescyclus eindigt wanneer de BL-spanning terug naar de grond is getrokken. Figure \ref{fig:leescyclus} illustreert de hele leescyclus.

\begin{figure}[!ht]
  \centering
  \includegraphics[width=0.9\textwidth]{../fig/hfdstk-timing-leescyclus.png}
  \caption[Leescyclus]{Leescyclus}
  \label{fig:leescyclus}
\end{figure} 

\subsection{Vergelijking van de geheugenconfiguraties}
Er werden 20 mogelijke geheugenconfiguraties geselecteerd als kandidaat voor het finale ontwerp, hun positie in de evaluatieruimte wordt getoond in figuur \ref{fig:final20all1}. Hierop staat het energieverbruik op de x-as, delay op de y-as en de oppervlakte wordt weergegeven door de grootte van de bolletjes. Het aantal effectief gebruikte referentiecellen wordt constant gehouden voor de verschillende configuraties. Dit wordt gedaan om het energieverbruik te verkleinen en omdat men maar een beperkt aantal cellen nodig heeft om een goede referentiedistributie te verkrijgen. De delay wordt voornamelijk bepaald door het opladen van de bitlijnen, die ook de grootste energieverbruikers zijn. De snelheid van de bitlijnen wordt dan weer bepaald door het aantal woordlijnen, dit kan gezien worden in figuur \ref{fig:final20all2}. Het aantal bitlijnen beinvloedt dan weer meer het energieverbruik. Dit extra energieverbruik gaat in de eerste plaats naar de WL-buffers, in mindere mate naar de bitlijn decoders en nog minder naar de bitlijn zelf. Bij alle geheugenconfiguraties komt het vermogenverbruik voornamelijk van de geheugecel, vervolgens in dalende lijn van de logica, de buffers en de sense amplifiers. De oppervlakte wordt bepaald door het aantal global blocks en de grootte van de decoders. Een groot aantal woordlijnen in combinatie met een klein aantal bitlijnen geeft de noodzaak aan een groot aantal global blocks, dit heeft een groot oppervlak als gevolg.\\
Er kan dus besloten worden dat een geheugen configuratie bestaande uit een klein aantal woordlijnen en evenveel bitlijnen een optimum zal geven voor energieverbruik en delay, maar een suboptimum voor oppervlakte.

\afterpage{
\begin{figure}[!ht]
  \centering
  \includegraphics[scale=0.45, angle = 90]{../fig/hfdstk-timing-all-sol1.png}
  \caption[Delay, energieverbruik en oppervlakte van alle geheugenconfiguraties]{Delay, energieverbruik en oppervlakte van verschillende geheugenconfiguraties van 1Mbit. Kleinste transistor-gate-oppervlakte is $0.009mm^{2}$, grootste transistor-gate-oppervlakte is $0.68mm^{2}$}
  \label{fig:final20all1}
\end{figure} 
\clearpage
}

\begin{figure}[!ht]
  \centering
  \includegraphics[scale=0.6]{../fig/hfdstk-timing-all-sol2.png}
  \caption[Delay, energieverbruik en oppervlakte van alle geheugenconfiguraties]{Invloed \#BL en \#WL op delay, energieverbruik en oppervlakte voor verschillende geheugenconfiguraties van 1Mbit. Kleinste transistor-gate-oppervlakte is $0.009mm^{2}$, grootste transistor-gate-oppervlakte is $0.68mm^{2}$}
  \label{fig:final20all2}
\end{figure} 

\section{Besluit}
In dit hoofdstuk werd de timing van alle logica in de geheugenarchitectuur in kaart gebracht. Hierbij werd er gekeken naar wat de gewenste opeenvolging van signalen is en hoe dit problemen of beperkingen in de architectuur kan teweegbrengen. Vervolgens werd met deze kennis een aantal geheugenconfiguraties ontworpen en vergeleken. De conclusie is dat een kleiner aantal woordlijnen en bitlijnen een optimale snelheid en energieverbruik voor een GB geven. Op het vlak van oppervlakte leidt dit tot een suboptimum.

\chapter{Volledige ontwerp}
\label{final}
\section{Het finaal ontwerp}
Het finale ontwerp is een geheugen dat uit 32BL, 32WL en 512GB bestaat. Van de 32 BL worden er 16 gebruikt voor het genereren van het referentiespanning. En van de 16 gebruikte cellen zijn er 6 in HRS en 10 en LRS. Dit om de referentiespanningsverdeling beter te centreren tussen de BL-spanningen voor cellen in RHS en LRS. De afmetingen van alle transistoren staan in tabel \ref{tab:transsize}. \\
Op dit geheugen wordt een speed-vdd-test uitgevoerd. Dit is een test waarbij de voedingspanning wordt verlaagd en er vervolgens geverifieerd wordt aan welke snelheid de leescyclus nog kan uitgevoerd worden. Hierbij wordt de dutycycle spanningsdeling-latching manueel gekozen op basis van het circuitgedrag op de voedingspanning. Dit geeft natuurlijk een meer optimistisch beeld dan dat er een kloksignaal met een bepaalde frequentie zou verwerkt worden door digitale logica om deze dutycycle te bekomen. De delay van de logica zou overigens niet lineair schalen met de voedingspanning, waardoor de dutycycle sowieso niet constant blijft. Figuur \ref{fig:speedvdd} toont de resultaten van deze test. Voor elk vakje in de shmoo plot werden 100 Monte Carlo simulaties uitgevoerd, een groen vakje stelt 100 geslaagde leesoperaties voor, bij een rood vakje is er minstens 1 leescyclus foutief verlopen. Zoals men duidelijk kan zien  daalt de leessnelheid bij het verlagen van de voedingspanning. Dit komt door een combinatie van 2 factoren. Ten eerste gaat de logica trager worden, dit heeft als gevolg dat de spanningsdeling later wordt uitgevoerd na het schakelen van de controlesignalen. Bovendien komt het signaal aan de gate van de ChargeBL- en DischargeSL-transistor eerder aan dan het WL-signaal. Hierdoor verschijnt er een overshoot bij het laden van de bitlijn zoals in het vorig hoofdstuk werd geillustreerd in figuur \ref{fig:critisch_timing1}. De overshoot treedt eerder op voor LRS-cellen omdat de BL hierbij minder lang moet opladen. Door de overshoot van de BL-spanning moet men langer wachten vooraleer de sense amplifier mag aangezet worden. De tweede factor die de leessnelheid doet vertragen is de SA zelf. Bij een voedingspanning van 1V kan deze binnen de 0.25ns schakelen. Bij lagere voedingspanningen kan dit afhankelijk van de mismatch tot 2ns duren vooraleer de sense amplifier gelatcht heeft.

\begin{table}
\begin{center}
\begin{tabularx}{\textwidth}{XXX}
\hline
Transistor & L (nm) & W (nm)\\
\hline
ChargeBL & 195 & 300 \\
DischargeBL & 45 & 100 \\
DischargeSL & 45 & 500 \\
Sa enableP & 45 & 900 \\
Sa enableN & 45 & 500 \\
Sa P & 45 & 1700 \\
Sa N & 45 & 1500 \\
Mux LB & 45 & 200 \\
Mux GB & 45 & 100 \\
\hline
\end{tabularx}
\end{center}
\caption[Transistor afmetingen eind ontwerp]{Afbeeldingen van de transistoren in het eind ontwerp}
\label{tab:transsize}
\end{table}
	
\begin{figure}[!ht]
  \centering
  \includegraphics[scale=0.8]{../fig/hfdst-final-vddspeed.png}
  \caption[Resultaten speed-vdd test]{Resultaten speed-vdd test}
  \label{fig:speedvdd}
\end{figure}

Verder kan men ook zien dat de schakeling een voedingspanning hoger of gelijk aan 0.8V nodig heeft om correct te kunnen werken. De verklaring hiervoor kan gezien worden in figuur \ref{fig:vblvdd}. Deze figuur stelt de distributie voor van de BL-spanningen van een cel in RHS, een cel in LRS en de referentiespanning in functie van verschillende voedingspanningen. Er kan duidelijk gezien worden dat bij het verlagen van de voedingspanningen deze distributies dichter bij elkaar komen te liggen en dat een voedingspanning van 0.8V wel degelijk een limiet is. Aangezien de extrema's van de distributies bij een voedingspanning van 0.8V zo dicht bij elkaar zitten, wordt er verwacht dat de schakeling occasioneel zal falen omdat de SA ontworpen is voor een $\Delta V$ van 35mV. Dit is echter niet gebeurd bij de 100 Monte Carlo simulaties.
Als men naar de distributies kijkt voor een voedingspanning van 1V zal men opmerken dat deze niet dezelfde zijn als de distributies getoond in hoofdstuk \ref{loadanalysis}(figuur \ref{fig:distswitch}). De reden hiervoor is dat men in de speed-vdd-test niet wacht to de bitlijn volledig is opgeladen, wat een tijdswinst oplevert. Ook werd er in hoofdstuk \ref{loadanalysis} gesuggereerd dat een energiewinst zou bereikt kunnen worden door een andere last te kiezen (sectie \ref{anderelast}). Hoewel dit mogelijk is, heeft dit wel als nadeel dat de schakeling minder tolerant is voor voedingspanningvariaties. Tenslotte moet ook vermeld worden dat de speed-vdd-test uitgevoerd werd op een (SPICE) temperatuur van $30^{\circ}\mathrm{C}$. Moest deze schakeling worden geïmplementeerd in een processor is de kans groot dat dit onderhevig zal zijn aan temperaturen tussen de $30^{\circ}\mathrm{C}$ en $60^{\circ}\mathrm{C}$ wat ook tragere leessnelheden zal opleveren. Hoe traag werd echter niet onderzocht.

\begin{figure}[!ht]
  \centering
  \includegraphics[scale=0.8]{../fig/hfdst-final-vddbl.png}
  \caption[Bl-spanningen i.f.v. Vdd]{Bl-spanningen i.f.v. Vdd}
  \label{fig:vblvdd}
\end{figure}

Het totale energieverbruik van een leescyclus bij een voedingspanning van 1V is gemiddeld 0.51pJ. Hierbij gaat 25\% van de energie naar de logica, 2\% naar de sense amplifier, 65\% naar de stroomdeling en 8\% naar de buffers. Hierbij werden de decoderbuffers bij logica gerekend.

\section{Vergelijking met de literatuur}
Het vergelijken van 2 chips is geen evidentie. Ten eerste verschillen vaak de vooropgestelde chipspecificaties, ten tweede verschillen vaak de technologieën en tenslotte geven veel papers niet alle resultaten weer om goed te kunnen vergelijken. Chipspecificaties hangen af van de noden van de applicatie van de chip. Voor automotive applicaties bijvoorbeeld is er vooral nood aan geheugens die bij hoge temperaturen een hoge betrouwbaarheid hebben. Medische toepassingen daarentegen hebben  nood aan low-power chips. Onder verschillende technologieën kan er een onderscheid gemaakt worden tussen de technologie van de logica en van het geheugen. Zo wordt er vaak voor verschillende toepassingen een andere soort NOR-flash geheugencel gebruikt: charge-trapping-cellen worden gebruikt voor betere betrouwbaarheid, split-gates-cellen voor hoge performantie \cite{5783209}. De schakeling ontworpen in dit werk kan bij de snellere geheugens worden gecategoriseerd vergeleken met NOR-geheugens in de industrie \cite{6649105}\cite{4433985}\cite{4027813}. Hierbij werd er gekeken naar de random-access-leessnelheid. Vaak kan een groot verschil in leessnelheid (gedeeltelijk) verklaard worden door de bitlijncapaciteit. Vermits het energieverbruik berekend kan worden d.m.v/ $CV_{vdd}^{2}$ kan men vermoeden dat de schakeling in dit werk ook bij de meer energiezuinige schakelingen hoort. De meeste schakelingen gevonden in de literatuur hebben namelijk een hogere voedingsspanning. De werking van het ontworpen RRAM geheugen op verschillende temperaturen werd in dit werk niet onderzocht. Naast verschillen in de werking van de logica zal ook de memristor onderhevig zijn aan temperatuursveranderingen. Volgens \cite{5948374} zal bij een $HfO_{2}$ geheugencel de $ROFF/RON$-verhouding dalen bij stijgende temperatuur. Ondanks deze daling in performatie ziet men in de industrie toch RRAM-chips opduiken die functioneren bij hoge temperaturen\footnote{http://www.crossbar-inc.com/markets/automotive.html}. Men kan besluiten dat met de afbakening in het achterhoofd de schakeling een goede prestatie levert t.o.v. schakelingen in de literatuur.

\section{Besluit}
Een finale schakeling werd ontworpen en geëvalueerd. Hierbij werd er voornamelijk gekeken naar de prestatie onder verschillende voedingspanningen. Er werd een absolute limiet van 0.8V gevonden voor de voedingspanning en verklaard. Verder werd er vergeleken met NOR-flash schakelingen in de literatuur en werd er besloten dat de schakeling in dit werk een goede concurrent is op het vlak van snelheid en energieverbruik. Andere aspecten kunnen niet vergeleken worden aangezien deze niet onderzocht werden in dit werk.
\chapter{Besluit}
\label{besluit}
In dit werk werd een RRAM leescircuit ontworpen. De 1T1R-cel die de informatie bevat bestaat uit een minimale transistor en een resistief geheugenelement, de memristor (\textbf{hoofdstuk \ref{cell}}). Voor leessimulaties werden de geheugenelementen gemodeleerd als weerstanden waarvan de weerstandswaardes gebaseerd zijn op hafniumoxidememristoren.\\\\
Cellen worden in geheugenmatrices gegroepeerd (\textbf{hoofstuk \ref{architecture}}), aan deze matrices worden decoders en passgates toegevoegd die samen een local block (LB) vormen. Een local block kan aan diens uitgang zowel een datasignaal leveren als een referentiesignaal. De uitgangen van 2 LBs vormen de ingangs- en referentiespanning van een sense amplifier; de combinatie van twee LBs en een SA heet global block. Data- en referentiesignalen worden verkregen via een spanningsdeling met een bepaalde lastimpedantie en celimpedantie. Voor het datasignaal vindt er een spanningsdeling plaats op één BL, deze BL-spanning wordt naar de uitgang van het LB overgebracht door de bijhorende passgate te activeren. Voor het referentiesignaal worden er op meerdere BLs spanningsdelingen uitgevoerd, de referentiespanning wordt gevormd door de BLs kort te sluiten aan de uitgang met de passgates. Door meerdere referentiecellen te gebruiken kan de distributie van het referentiesignaal gemanipuleerd worden.\\\\
Om een zo groot mogelijk verschil tussen data- en referentiespanning te bekomen blijkt er een optimale lastimpedantie (\textbf{hoofdstuk \ref{loadanalysis}}). Verschillende topologieën van impedanties zijn onderzocht naar BL-spanningsverschil, snelheid en spanningsval over geheugenelement, alsook de invloed van variabiliteit hierop. Omwille van dit laatste is het niet mogelijk om met transistoren met minimale lengtes te werken. Een enkele transistor met niet-minimale afmetingen die gebruikt wordt als schakelaar blijkt er als beste uit te komen.\\\\
De sense amplifier moet het kleine spanningsverschil tussen data- en referentiesignaal correct versterken tot de voedingsspanning (\textbf{hoofdstuk \ref{sensamp}}). De gebruikte topologie voor de SA in dit werk is de drain-input latch-type SA. De belangrijkste eigenschap van een SA wat correcte werking betreft is de offsetspanning. De distributie hiervan wordt in kaart gebracht met sensitiviteitsanalyses. Door korte overlap tussen passgate- en SA-enable-signaal kan de spreiding van de offsetspanning kleiner gemaakt worden. Het RC-latch-effect zorgt ervoor dat de SA gedurende de snelle fase van de versterking geen invloed merkt van de grote BL-capaciteit, waaraan die blootgesteld wordt tijdens de overlap. Op basis van een lineaire sweep van transistorafmetingen werden pareto-optimale SAs bepaald wat snelheid, dynamische energie en offsetspanning betreft.\\\\
Naast lastimpedantie en SAs is er in het geheugen ook nood aan omringende logica zoals buffers, passgates, decoders,... Deze werden onderzocht in \textbf{hoofdstuk \ref{periphery}}.\\\\
\textbf{Hoofdstuk \ref{timing-optimization}} brengt de timing van alle signalen in het geheugen in kaart. Hierbij werd er gekeken welke beperkingen er opgelegd moeten worden aan de architectuur om een juiste timing te hebben. Met deze kennis worden een aantal geheugens ontworpen van 1Mbit en met elkaar vergeleken op vlak van snelheid, energieverbruik en oppervlaktegebruik.\\\\
Uiteindelijk wordt een geheugenarchitectuur met 32 woordlijnen, 32 bitlijnen en 512 global blocks gekozen als eindontwerp. Hierop wordt er een speed-vdd-test (\textbf{hoofdstuk \ref{final}}) uitgevoerd en wordt de prestatie van het circuit vergeleken met de literatuur. Men kan besluiten dat met de afbakeningen en beperkingen in het achterhoofd de schakeling een goede prestatie levert t.o.v. schakelingen in de literatuur.


% Indien er bijlagen zijn:
\appendixpage*          % indien gewenst
\appendix
\chapter{Ladingsinjectie bij het gebruik van ideale SPICE bronnen.}
\label{app:chargeinj}
Bij het berekenen van de energie consumptie van verschillende bouwblokken, valt het op dat de voeding stroom opneemt i.p.v. levert bij het schakelen van de ideale spice bronnen. Dit komt door een ladingsinjectie van de ideale spice bronnen. Om dit te verifieren, worden de stromen van een simpele inverterschakeling bestudeerd. Figuur \ref{fig:chargeinj_inv} stelt een simple inverterschakeling voor met relevante parasitaire capaciteiten. Bij het aanleggen van een stapfunctie aan de inverter vloeit er een stroom door de capaciteiten. Omdat de positieve stroom die geobserveerd wordt in de voeding afkomstig is van de ingang, moet de som van de stromen door de ingang, voeding en grond nul zijn. De stroom door de ingang kan in SPICE opgemeten worden door een weerstand met resistieve waarde gelijk aan nul, in serie met de ingang te zetten.\\
Figure \ref{fig:chargeinj_cur} toont deze drie stromen. In het eerste deel van de figuur (tot tijdstip 30ps) is de spanning aan de input al aan het stijgen maar de invertor is nog niet aangeschakeld. De voedings- en grondstromen komen dan puur van de ingang. Vanaf tijdstip 30ps, is de invertor aan het schakelen en is er een aandeel van de stroom in de grond die afkomstig van de voeding is. De som van de drie stromen is ten alle tijden gelijk aan nul.
We kunnen dus besluiten dat er een ladingsinjectie is van ideale spice bronnen in het circuit. Dit heeft een invloed op de stromen en daardoor ook op de energie berekeningen, maar dit verwaarlozen we bij onze berekeningen.

\begin{figure}[!ht]
  \centering
  \includegraphics[width=0.4\textwidth]{../fig/hfdst-chargeinj-inv.png}
  \caption[Ladingsinjectie: testcircuit]{Testcircuit ladingsinjectie}
  \label{fig:chargeinj_inv}
\end{figure}

\begin{figure}[!ht]
  \centering
  \includegraphics[width=\textwidth]{../fig/hfdst-chargeinj-currents.png}
  \caption[Ladingsinjectie: stroom]{Stromen in circuit}
  \label{fig:chargeinj_cur}
\end{figure}


\backmatter
% Na de bijlagen plaatst men nog de bibliografie.
% Je kan de  standaard "abbrv" bibliografiestijl vervangen door een andere.
\bibliographystyle{abbrv}
\bibliography{referenties}

\end{document}

%%% Local Variables: 
%%% mode: latex
%%% TeX-master: t
%%% End: 
