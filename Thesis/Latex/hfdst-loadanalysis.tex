\chapter{Lastimpedantie-analyse}
\label{loadanalysis}
Om een cel uit te lezen wordt er een spanning gegenereerd op de bitline door middel van spanningsdeling.
Het is dus belangrijk om de 2 impedanties van de spanningsdeler zodanig te kiezen voor optimale snelheid, bitline spanningsverschil en memristorretentie.
Ook belangrijk is dat deze impedanties robuust zijn tegen variabiliteit.

\section{Impedantiekandidaten}

\subsection{Schakelaarsbelasting}

\subsection{Extra MOS-transistor}

\subsection{Diode-geconnecteerde MOS-transistor}

\subsection{Bulk-geconnecteerde MOS-transistor}

\section{Nominaal gedrag}

\section{Gedrag onder variabiliteit}

\subsection{Monte Carlo simulaties}

\section{Besluit}

