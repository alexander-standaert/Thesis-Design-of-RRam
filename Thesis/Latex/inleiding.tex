\chapter{Inleiding}
\label{inleiding}

Vandaag de dag is elektronica niet meer uit het leven weg te denken. Van de smartphone tot het digitaal horloge, van de boordcomputer in de moderne wagen tot de microprocessor in de vaatwasser, overal vind je wel elektronica terug.
Sinds Gordon Moore ongeveer 50 jaar geleden de uitspraak deed dat het aantal transistoren op eenzelfde oppervlakte per twee jaar zou verdubbelen \cite{Moo65}, is de industrie er over het algemeen goed in geslaagd dit te verwezenlijken. Dit leidde tot de snelle en uiterst complexe chips die we vandaag allemaal goedkoop aankopen.

Naarmate de processorkracht groter werd, steeg ook de vraag voor grotere en snellere geheugens om deze processorkracht ook effectief uit te buiten. Static Random Access Memory (SRAM) blijft een populaire keuze voor snelle ingebedde geheugens, maar heeft het nadeel vluchtig te zijn: eenmaal de voedingsspanning wordt afgeschakeld, verdwijnt de informatie. Flash-geheugens, door veel mensen gebruikt voor massa-opslag in USB-sticks of SSDs, hebben ook hun weg gevonden naar het ingebedde domein en behoren wel tot de klasse van niet-vluchtige geheugens.
Het blijkt echter bijzonder moeilijk om flash-geheugens verder te verkleinen \cite{Pra10}.

Onderzoek naar nieuwe geheugens is dan ook onontbeerlijk. Zo zijn er al nieuwe kandidaten in opmars die hoopgevend zijn om te concurreren met (ingebedde) flash-geheugens. MRAMs (Magnetic RAMs) en in het bijzonder STT-RAM (Spin-Transfer Torque) zullen op termijn een belangrijke rol gaan spelen.

Een andere kandidaat is Resistive RAM (RRAM of ReRAM). Daar waar SRAM- en flash-cellen informatie bevatten via het al dan niet aanwezig zijn van lading, bevat een RRAM-cel informatie door een bepaalde elektrische weerstandswaarde aan te nemen. RRAM zou geen problemen hebben om nog even op de klassieke manier mee te schalen en is dus zonder meer een interessante piste om verder te onderzoeken. Bovendien zou RRAM gefabriceerd kunnen worden met goedkopere processen dan flash-geheugens: bij flash-geheugenfabricatie zijn vaak dure extra maskers vereist terwijl RRAM-fabricatie geïntegreerd kan worden met een standaard CMOS-productieproces.

\section{Doel en afbakening van dit werk}
Dit werk beschrijft het ontwerp van een 1Mbit RRAM-geheugen voor ingebedde toepassingen. De doelstelling is een pareto-optimaal circuit te ontwerpen. De pareto-doelstellingen zijn snelheid, dynamische energie en oppervlakte. Het ontwerp is ook gewapend tegen variabiliteit d.w.z. ongecorreleerde gedragsvariaties van componenten. De analyse focust op de leesbewerking, de schrijfbewerking valt buiten het bereik van dit werk. Er worden wel mogelijke oplossingen aangereikt, maar deze werden niet uitdrukkelijk onderzocht. Bij het ontwerp is ook aandacht besteed aan het vermijden van destructieve leescycli.

Voor de leesbewerking wordt het geheugenelement gemodeleerd als een weerstand waarvan de weerstandswaarde afhangt van de celtoestand. Om variabiliteit te onderzoeken, worden Monte Carlo simulaties uitgevoerd waarbij de weerstandswaarde een Gaussisch verdeelde variabele is.

Temperatuursvariaties werden niet systematisch onderzocht, maar aangezien temperatuur een globale variabele is en het systeem differentieel werkt, wordt niet verwacht dat de performantie aanzienlijk zal verminderen.

Alle analyses in dit werk zijn uitgevoerd met Spectre simulaties met 45nm PTM transistormodellen. In tabel \ref{tab:properties} zijn technologieparameters te zien, waarvan meermaals gebruik gemaakt wordt doorheen dit werk.

\begin{table}
	\begin{tabular}{ccc}
	\hline
    $A_{\beta n}$ & 2 $\% \mu m$ & $\beta$-Pelgrom constante voor nMOS-transistoren \\
    $A_{\beta p}$ & 1,2 $\% \mu m$ & $\beta$-Pelgrom constante voor pMOS-transistoren \\
    $A_{V_{T} n}$ & 2,82 $mV \mu m$ & VT-Pelgrom constante voor nMOS-transistoren \\
    $A_{V_{T} n}$ & 2,5 $mV \mu m$ & VT-Pelgrom constante voor pMOS-transistoren \\
    $C_{WL}$ & 0,18 fF/cel & WL-capaciteit, stijgt lineair met het aantal cellen eraan \\
    $C_{BL}$ & 0,18 fF/cel & BL-capaciteit, stijgt lineair met het aantal cellen eraan \\
    $C_{inv}$ & 0,35 fF & intrinsieke capaciteit van een CMOS inverter \\
    $\mu_{HRS}$ & 32500 $\Omega$ & verwachtingswaarde van een HRS geheugenelement \\
    $\mu_{LRS}$ & 7500 $\Omega$ & verwachtingswaarde van een LRS geheugenelement \\
    $\sigma_{HRS}$ & 833 $\Omega$ & standaarddeviatie van een HRS geheugenelement \\
    $\sigma_{LRS}$ & 833 $\Omega$ & standaarddeviatie van een LRS geheugenelement \\
    $V_{DD}$ & 1 V & voedingsspanning \\
    $V_{SS}$ & 0 V & grondspanning \\
    \hline
  \end{tabular}
  \caption[Technologieparameters]{Numerieke technologieparameters waarvan gebruik gemaakt is in simulaties}
  \label{tab:properties}
\end{table}


\section{Structuur van de tekst}
In hoofdstuk \ref{cell} wordt de technologie van een RRAM geheugen uiteengezet, alsook diens toepassingen. In hoofdstuk \ref{architecture} wordt het geheugensysteem vanuit vogelperspectief besproken. Er wordt hier ook aangehaald wat de regelbare parameters zijn van de architectuur. Voor een robuuste, snelle en laag-energetische leesoperatie is het belangrijk het geheugenelement te combineren met een zorgvuldig gekozen impedantie, dit wordt onderzocht in hoofdstuk \ref{loadanalysis}. Uiteindelijk worden bits afgeleverd aan de uitgang van het systeem: de sense amplifier zorgt hiervoor en wordt besproken in hoofdstuk \ref{sensamp}.
In de geheugenstructuur worden ook bouwblokken zoals decoders, buffers en passgates gebruikt om op basis van het opgegeven adres de juiste cel aan te spreken; deze worden beschreven en geanalyseerd in hoofdstuk \ref{periphery}.
In hoofdstuk \ref{timing-optimization} wordt de timing van controlesignalen onderzocht alsook de optimalisatie van het systeem door de architectuurparameters te tunen. Tenslotte wordt een overzicht gegeven van de resultaten van het volledige ontwerp in hoofdstuk \ref{final}.
