\chapter{Sense Amplifier analyse}
\label{sensamp}
Een sense amplifier versterkt kleine signaalverschillen tot rail-tot-rail signalen. Aangezien de uitgangsignalen hiervan ook de uitgelezen bits zijn van het geheugen, is het uiterst belangrijk dat dit op een correcte manier gebeurt, ondanks alle mogelijke nadelige gevolen van variabiliteit.
In dit hoofdstuk wordt de sense amplifier dus ook wat dieper onderzocht. Uiteindelijk wordt een SA ontwerpen die functioneert op een robuuste manier en die ook voldoende snel en laagenergetisch is.

\section{Types SA}
Er bestaan heel wat verschillende soorten sense amplifiers, traditioneel kunnen die op twee verschillende manieren geclassificeerd worden. Enerzijds kan men onderscheid maken tussen differentiële en niet-differentiële SA. Anderzijds kan men een opdeling maken in voltage en current SA. Een voorbeeld van een niet-differentiële SA is een inverter. De drempelspanning waarbij de inverter schakelt wordt bepaald door de groottes van de transistoren. Deze architectuur is echter bijzonder susceptibel aan globale variabelen zoals temperatuur en voedingsspanning. Een dergelijk probleem kan opgelost worden door een referentiesignaal te voorzien dat gelijke variaties ervaart als het datasignaal, als gevolg van deze globale schommelingen. Voor een dergelijke architectuur zijn uiteraard differentiële SAs nodig. Current en voltage SA verschillen in ingangsimpedantie: bij een hoge ingangsimpedantie spreekt men van voltage-sensing, bij een lage impedantie spreekt men van current-sensing. Voorbeelden van voltage- en current-mode sense amplifiers zijn geïllustreerd in figuren \ref{fig:voltagemodesa} en \ref{fig:currentmodesa}. Voor de architectuur van dit werk waarbij de data- en referentiesignalen opgewekt worden door een spanningsdeling met eigen gekozen lastimpedantie, is een SA nodig met een grote ingangsimpedantie.

\begin{figure}
  \centering
  \includegraphics[width=\textwidth]{../fig/hfdstk-sensamp-voltagemode.png}
  \caption[Voltage mode sense amplifiers]{Voltage mode sense amplifiers}
  \label{fig:voltagemodesa}
\end{figure}
\begin{figure}
  \centering
  \includegraphics[width=\textwidth]{../fig/hfdstk-sensamp-currentmode.png}
  \caption[Current mode sense amplifiers]{Current mode sense amplifiers, reproduced from\cite{5548588}}
  \label{fig:currentmodesa}
\end{figure}

\begin{figure}
  \centering
  \includegraphics[scale=0.4]{../fig/hfdstk-sensamp-ourSA.png}
  \caption[De drain-input latch-type SA]{De drain-input latch-type SA}
  \label{fig:ourSA}
\end{figure}

Omwille van eenvoud en goede performantie\cite{Cos09} wordt er in wat volgt voortgewerkt met de zogenoemde drain-input latch-type SA van figuur \ref{fig:ourSA}.


\section{Offsetspanning}
Een ideale sense amplifier zal voor elke twee ingangssignalen correct versterken, tenzij deze signalen dezelfde zijn. Dit is een conditie die de SA in een metastabiele toestand brengt. In de praktijk bestaat er echter wegens variabiliteit een onderlimiet voor het ingangsspanningsverschil waarbij er correct versterkt wordt. Deze limiet heet de offsetspanning en wordt geïllustreerd in figuur \ref{fig:offset}. De offsetspanning van een SA is in de ontwerpfase een stochastische variabele met gemiddelde 0V, pas nadat een chip gefabriceerd is ligt de offsetspanning definitief vast [al kan het zijn dat deze met de tijd nog verandert].\\
Er zijn 2 manieren waarop men de offsetspanning van een systeem kan aanpakken: ofwel ontwerpt men het systeem zodanig dat het verschil van de ingangssignalen van de SA groot genoeg is zodat het [in 99,x\% van de gevallen] groter is dan de offsetspanning, ofwel bouwt men een mechanisme in waarbij de offsetspanning na fabricatie gemeten en vervolgens gecompenseerd wordt. In dit werk is gekozen voor de eerste oplossing.
Hiervoor is het wel belangrijk te onderzoeken wat de verdeling is van de offsetspanning, dit wordt gedaan in de volgende sectie.

\begin{figure}
  \centering
  \includegraphics[scale=0.4]{../fig/hfdstk-sensamp-offset.png}
  \caption[SA offsetspanning]{Illustratie van offsetspanning}
  \label{fig:offset}
\end{figure}

\section{Sensitiviteitsanalyse}
De SA wordt gerealiseerd als een circuit bestaande uit transistors. Elke transistor wordt in onze simulaties gekarakteriseerd door 2 stochastische parameters met een normale verdeling, nl. $\Delta V\textsubscript{t}$ en $\Delta \beta$. De spreiding van deze verdelingen is gekend: $\sigma_{\Delta V_{t}} = \frac{A\textsubscript{V\textsubscript{t}}}{\sqrt{W L}}$ en $\sigma_\frac{{\Delta \beta}}{\beta} = \frac{A_{\beta}}{\sqrt{W L}}$. Met een sensitiveitsanalyse kan men uit deze standaardafwijkingen de standaardafwijking van de offsetspanning $\sigma_{V_{offset}}$ berekenen. Hierbij wordt verondersteld dat de stochastische variabele $V_{offset}$ een lineaire combinatie is van de normaal verdeelde afwijkingen $(\Delta V_{t})_{i}$ en $(\frac{\Delta_{\beta}}{\beta})_{i}$: $V_{offset}=\sum\limits_{i=1}^{N} a_{i} (\Delta V_{t})_{i} + b_{i} (\frac{\Delta_{\beta}}{\beta})_{i}$.
$a_{i}$ en $b_{i}$ zijn de gevoeligheden van de offset naar de variatieparameters: $a_{i}=\frac{\partial V_{offset}}{\partial (\Delta V_{t})_{i}}$ en $b_{i}=\frac{\partial V_{offset}}{\partial (\frac{\Delta_{\beta}}{\beta})_{i}}$.
Voor een dergelijke variabele geldt dan: $\sigma_{V_{offset}}=\sqrt{\sum\limits_{i=1}^{N} a_{i}^{2} (\sigma_{\Delta V_{t}})_{i}^{2} + b_{i}^{2} (\sigma_{\frac{\Delta_{\beta}}{\beta}})_{i}^{2}}$.

Er moet wel geverifieerd worden of de stelling dat er een lineaire afhankelijkheid is tussen $V_{offset}$ en de variatieparameters gegrond is.
Dit kan gedaan worden aan de hand van een analyse waarbij elke variatieparameter afzonderlijk gesweept wordt, een noodzakelijke maar niet voloende vereiste.

\subsection{Sensitiviteitsanalyse op een minimale SA}
In figuur \ref{fig:min-sensanalysis} wordt het resultaat getoond voor een dergelijke analyse bij een SA met minimale afmetingen. Op deze grafiek kan gezien worden dat de offsetspanning lineair varieert met de variatieparameters en dat er dus voldaan is aan bovenstaande vereiste. Merk op dat de richtingscoëfficient van deze curves gelijk is aan $a_{i} (\Delta V_{t})_{i}$ en $b_{i} (\frac{\Delta_{\beta}}{\beta})_{i}$. In tabel \ref{tab:min-sensanalysis} worden de resultaten en de resulterende standaardvariatie van de SA getoond.
Er moet opgemerkt worden dat er bij deze simulatie slechts gesweept werd voor de variatieparameters van -4$\sigma$ tot 4 $\sigma$. Dit is omwille van het feit dat voor de minimale transistoren de standaardvariatie het grootst is. In de Spectre-simulaties zouden transistoren voor te grote negatieve $\beta$-mismatch stroom leveren in de omgekeerde richting. Deze situatie zal fysisch nooit optreden.\\\\

\begin{figure}[!ht]
\centering
\subfloat[Vt-mismatch]{ \includegraphics[width=0.9\textwidth] {../fig/hfdstk-sensamp-vt-sensitivity.png}} \\
\subfloat[$\beta$-mismatch]{ \includegraphics[width=0.9\textwidth] {../fig/hfdstk-sensamp-b-sensitivity.png}}

\caption[Sensitiviteitsresultaten: offsetspanning i.f.v. mismatchvariabelen]{Sensitiviteitsresultaten: offsetspanning i.f.v. mismatchvariabelen}
\label{fig:min-sensanalysis}
\end{figure}


\begin{table}
\begin{tabular}{cccccc}
\hline 
Transistor & Parameter & Richtingscoëfficiënt [$\frac{mV}{\sigma}$] & W [nm] & L [nm] & $\sigma$ \\ 
\hline 
Mupbar & Vt & 22.7 & 100 & 45 & 37.3mV \\ 
Mup & Vt & -22.3 & 100 & 45 & 37.3mV \\ 
Mupbar & $\beta$ & 13.6 & 100 & 45 & 17.9\% \\ 
Mpassn & $\beta$ & 13.5 & 100 & 45 & 29.8\% \\ 
Mpassbarn & $\beta$ & -13.1 & 100 & 45 & 29.8\% \\ 
Mup & $\beta$ & -13.0 & 100 & 45 & 17.9\% \\ 
Mdownbar & $\beta$ & -9.4 & 100 & 45 & 29.8\% \\ 
Mdown & Vt & -9.3 & 100 & 45 & 42.0mV \\ 
Mdownbar & Vt & 9.2 & 100 & 45 & 42.0mV \\ 
Mdown & $\beta$ & 8.2 & 100 & 45 & 29.8\% \\ 
Mpassp & $\beta$ & -4.5 & 100 & 45 & 17.9\% \\ 
Mpassbarp & $\beta$ & 4.4 & 100 & 45 & 17.9\% \\ 
Mpassbarp & Vt & 0.70 & 100 & 45 & 37.3mV \\ 
Mpassp & Vt & -0.70 & 100 & 45 & 37.3mV \\ 
Mbottom & $\beta$ & 0.083 & 100 & 45 & 29.8\% \\ 
Mbottom & Vt & -0.033 & 100 & 45 & 42.0mV \\ 
Mpassbarn & Vt & 0 & 100 & 45 & 42.0mV \\
Mpassn & Vt & 0 & 100 & 45 & 42.0mV \\
Mtop & Vt & 0 & 100 & 45 & 37.3mV \\
Mtop & $\beta$ & 0 & 100 & 45 & 17.9\% \\
\hline 
\hline & $\sigma_{Voffset}$: & 45.7mV & & & \\
\hline
\end{tabular} 
\caption[Sensitiviteitsanalyse van de minimale SA]{Sensitiviteitsanalyse van de minimale SA}
\label{tab:min-sensanalysis}
\end{table}

Opmerkelijk bij deze analyse is dat er een significante bijdrage is van de passgates door $\beta$-mismatch. Een nadere observatie leert dat deze bijdrage optreedt door ladingsinjectie van de passgates die niet meer gematched is (zie figuur \ref{fig:chargeinjectionmismatch})\footnote{Voor de gebruike transistormodellen treedt in simulaties deze ladingsinjectievariatie op door $\beta$-variaties. In werkelijkheid komt de exacte ladingsinjectieverdeling allicht niet overeen met de simulaties. Maar het is desalniettemin aannemelijk dat er ladingsinjectiemismatch kan optreden door afmetingvariaties van transistoren.}.
Hierbij moet wel worden opgemerkt dat er voor deze simulatie geen overlap is tussen het controlesignaal om de passgate aan te zetten en het signaal om de SA te activeren (zie figuur \ref{fig:no-overlap}).
De reden hiervoor is dat als er overlap tussen deze signalen is, de SA ook de BL zou trachten op te laden. Hierbij zou men in eerste instantie verwachten dat er moet ingeboet worden aan snelheid en dat dit extra energie zou kosten.

Men kan argumenteren dat er een korte overlap zou kunnen toegelaten zijn, waarna er voldoende spanningsverschil tussen de 2 ingangs-uitgangsknopen zou opgebouwd zijn opdat de ladingsinjectie geen effect meer kan hebben op het eindresultaat (zie figuur \ref{fig:overlap}). Een tegenargument is dat de leescyclus door het opladen van de BL langer zou duren.
\begin{figure}
  \centering
  \includegraphics[scale=0.4]{../fig/hfdstk-sensamp-chargeinjectionmismatch.png}
  \caption[Foutief latchen door $\beta$-mismatch]{Door $\beta$-mismatch is ladingsinjectie van de passgates niet meer gematched en gaat de SA foutief latchen}
  \label{fig:chargeinjectionmismatch}
\end{figure}

\begin{figure}[!ht]
\centering
\subfloat[Transiënte simulatie zonder overlap]{ \includegraphics[width=0.5\textwidth] {../fig/hfdstk-sensamp-nooverlap.png} \label{fig:no-overlap}}
\subfloat[Transiënte simulatie met overlap]{ \includegraphics[width=0.5\textwidth] {../fig/hfdstk-sensamp-overlap.png} \label{fig:overlap}}
\caption[Transiënte simulatie zonder en met overlap]{Transiënte simulatie zonder en met overlap tussen passgate en SA operatie. Er zit geen variatie op de circuitelementen. De ingangs-uitgangsknopen van de SA zijn initieel opgeladen op 0V en 1V. De spanning op de LB-uitangsknopen bedraagt 0,4V en 0,55V.}\label{fig:no-overlap-overlap}
\end{figure}

\subsection{RC-latch-effect}
\label{RC-latch-effect}
De situatie waarbij er volledige overlap is tussen de controle signalen kan vereenvoudigd worden opgesteld met de situatie van figuur \ref{fig:RC-latch}. De 2 passgates (die van LB en SA) worden voorgesteld als een weerstand, de parasitaire capaciteit tussen de 2 passgates wordt verwaarloosd. CL bedraagt voor deze simulatie 46 fF, het equivalent voor een BL waaraan 256 cellen hangen. Cint bedraagt voor een SA met minimale transistorafmetingen 161 aF. Wanneer het dynamisch latch-gedrag bekeken wordt voor verschillende waardes van R, treedt er een merkwaardig effect op (zie figuur \ref{fig:RC-latch-sim}): voor voldoende grote waardes van R lijkt het alsof de grote capaciteit ontkoppeld is van de latch tot op een zeker tijdstip, waarna een veel tragere settling optreedt.
\begin{figure}
  \centering
  \includegraphics[scale=0.4]{../fig/hfdstk-sensamp-RC-latch.png}
  \caption[Simulatieopstelling voor het RC-latch-effect]{Simulatieopstelling voor het RC-latch-effect}
  \label{fig:RC-latch}
\end{figure}
\begin{figure}
  \centering
  \includegraphics[scale=0.4]{../fig/hfdstk-sensamp-RC-latch-sim.png}
  \caption[Simulatieresultaten voor het RC-latch-effect]{Simulatieresultaten voor het RC-latch-effect:de 2 ingangs-uitgangsknopen zijn voorgeladen op 400mV en 380mV. Na 1,6ns wordt de SA aangezet. De simulatie veronderstelt een SA waarbij er geen variaties zitten op de transistoren.}
  \label{fig:RC-latch-sim}
\end{figure}
De verklaring ligt in het feit dat CL zich voor hoge frequenties als een kortsluiting gedraagt (zie figuur \ref{fig:RC-latch-explain}), een plotse stroom vloeit door de weerstand waardoor er een spanningsval over de weerstand onstaat. Hierna bouwt er zich met een veel grotere tijdsconstante een spanning op over de capaciteit waardoor de ingangs-uitgansknopen volledig kunnen laden/ontladen tot VDD en VSS.
\begin{figure}
  \centering
  \includegraphics[width=\textwidth]{../fig/hfdstk-sensamp-RC-latch-explain.png}
  \caption[Invloed capaciteit op RC-latch effect]{Vergelijking situatie met voorgeladen (eindige) capaciteit en situatie met spanningsbron (oneindige capaciteit)}
  \label{fig:RC-latch-explain}
\end{figure}
Het gevolg van wanneer dit effect optreedt is dus dat het nuttige signaal zich snel - alsof er helemaal geen last aanhangt - en lineair opbouwt en dat er geen AC-signaal is over de BL-capaciteit. Een analyse van de respons van een RC-circuit op een lineair stijgende spanningsbron geeft meer duidelijkheid voor de voorwaarden waarop het RC-latch-effect optreedt (zie figuur \ref{fig:RC-latch-maplecircuit}). De respons van de spanning over de capaciteit is $Vcap(t) =  at - aRC(1-e^{-{\frac {t}{CR}}})$. Uit deze uitdrukking blijkt dat het RC-latch-effect optreedt wanneer de latch zonder last snel is (a<<1) en/of wanneer het RC-product hoog is (RC>>1).
Wanneer het effect zich voordoet zijn latching en RC-respons onafhankelijke processen. Wanneer de voorwaarden niet meer zo uitgesproken zijn, gaan deze processen met elkaar interfereren en is het moeilijk dit gecombineerde proces wiskundig te beschrijven.
\begin{figure}
  \centering
  \includegraphics[scale=0.4]{../fig/hfdstk-sensamp-RC-latch-maplecircuit.png}
  \caption[Circuit voor analyse voorwaarden RC-latch-effect]{Circuit voor analyse voorwaarden RC-latch-effect}
  \label{fig:RC-latch-maplecircuit}
\end{figure}

Conclusie van het RC-latch effect is dat de timing helemaal niet zo kritisch is: in theorie hoeft de overlap slechts even lang te duren als de delay van de SA wanneer er geen last op is aangesloten, maar het is niet erg als de overlap wat langer duurt.
De passgates mogen ook minimaal zijn, om hun aanweerstand te vergroten zodat het effect kan optreden.
In geval verder zou gewerkt worden met een SA zonder overlap met passgate-enable en SA-enable, zouden de passgates moeten geschaald worden om de mismatch te minimaliseren. Dit zou wel betekenen dat er per schakeling van de passgates een grotere hoeveelheid lading wordt geïnjecteerd.


\subsection{Sensitiviteitsanalyse voor minimale SA - vervolg}

In tabel \ref{tab:min-sensanalysis-overlap} worden de resultaten van een nieuwe sensitiveitsanalyse getoond voor een minimale SA, ditmaal waarbij er dus overlap is tussen passgate-enable en SA-enable. De spreiding van de offsetspanning is voor deze simulatieopstelling wel degelijk gedaald. Toch heeft de mismatch van de passgates nog steeds een significante bijdrage, dit kan verklaard worden a.d.h. van figuur \ref{fig:RC-latch}: de 2 weerstanden zijn niet gematcht, deze mismatch treedt op door zowel $\beta$- als $V_{T}$-mismatch van de passgates. De bijdrage van de mismatch van de differentiële paren daalt dan echter weer door de interactie met de passgates.

\begin{table}
\begin{tabular}{cccccc}
\hline 
Transistor & Parameter & Richtingscoëfficiënt [$\frac{mV}{\sigma}$] & W [nm] & L [nm] & $\sigma$ \\ 
\hline 
Mupbar & Vt & 15.3 & 100 & 45 & 37.3mV \\ 
Mup & Vt & -14.9 & 100 & 45 & 37.3mV \\ 
Mdownbar & Vt & 10.3 & 100 & 45 & 42.0mV \\ 
Mdown & Vt & -9.9 & 100 & 45 & 42.0mV \\
Mpassn & $\beta$ & 8.4 & 100 & 45 & 29.8\% \\ 
Mpassbarn & Vt & 6.8 & 100 & 45 & 42.0mV \\
Mpassbarn & $\beta$ & -6.8 & 100 & 45 & 29.8\% \\
Mpassn & Vt & -6.7 & 100 & 45 & 42.0mV \\ 
Mupbar & $\beta$ & 6.4 & 100 & 45 & 17.9\% \\ 
Mup & $\beta$ & -6.1 & 100 & 45 & 17.9\% \\ 
Mdownbar & $\beta$ & -5.6 & 100 & 45 & 29.8\% \\  
Mdown & $\beta$ & 5.2 & 100 & 45 & 29.8\% \\ 
Mpassp & Vt & 0.23 & 100 & 45 & 37.3mV \\
Mpassbarp & Vt & -0.23 & 100 & 45 & 37.3mV \\
Mtop & Vt & 0.17 & 100 & 45 & 37.3mV \\  
Mpassp & $\beta$ & -0.17 & 100 & 45 & 17.9\% \\ 
Mpassbarp & $\beta$ & 0.17 & 100 & 45 & 17.9\% \\ 
Mtop & $\beta$ & 0.12 & 100 & 45 & 17.9\% \\
Mbottom & $\beta$ & 0.050 & 100 & 45 & 29.8\% \\ 
Mbottom & Vt & 0.033 & 100 & 45 & 42.0mV \\ 
\hline 
\hline & $\sigma_{Voffset}$: & 31.7mV & & & \\
\hline
\end{tabular} 
\caption[Sensitiviteitsanalyse van de minimale SA met overlap tussen passenable en latchenable]{Sensitiviteitsanalyse van de minimale SA met overlap tussen passenable en latchenable}
\label{tab:min-sensanalysis-overlap}
\end{table}

\subsection{Sensitiviteitsanalyse voor gebruikte SA}

In tabel \ref{tab:ourSA-sensanalysis-overlap} worden de resultaten van de sensitiviteitsanalyse getoond voor de SA die gebruikt wordt in het finale geheugenontwerp. Deze is gekozen aan de hand van de resultaten van de paretosimulatie in de volgende sectie. Er is voor deze SA geopteerd voor overlap tussen passgate- en SA-operatie. De passgates zijn opgeschaald om $V_{T}$- en $\beta$-mismatch in te perken. Ondanks deze verkleinde weerstand, treedt het RC-latch-effect nog steeds op voor deze SA. Er dient tenslotte opgemerkt te worden dat een voldoende grote SA (passgates niet inbegrepen) het RC-latch effect niet nodig heeft om snel te latchen: een dergelijke SA kan genoeg stroom leveren om de BL-capaciteit snel op te laden.


\begin{table}
\begin{tabular}{cccccc}
\hline 
Transistor & Parameter & Richtingscoëfficiënt [$\frac{mV}{\sigma}$] & W [nm] & L [nm] & $\sigma$ \\ 
\hline 
Mup & Vt & -4.3 & 1700 & 45 & 9.0mV \\ 
Mupbar & Vt & 4.3 & 1700 & 45 & 9.0mV \\ 
Mpassn & Vt & -3.8 & 500 & 45 & 18.8mV \\
Mpassbarn & Vt & 3.7 & 500 & 45 & 18.8mV \\
Mpassbarn & $\beta$ & -3.0 & 500 & 45 & 13.3\% \\ 
Mpassn & $\beta$ & 3.0 & 500 & 45 & 13.3\% \\ 
Mup & $\beta$ & -1.8 & 1700 & 45 & 4.3\% \\ 
Mupbar & $\beta$ & 1.8 & 1700 & 45 & 4.3\% \\ 
Mdown & Vt & -1.1 & 1500 & 45 & 10.9mV \\ 
Mdownbar & Vt & 1.1 & 1500 & 45 & 10.9mV \\
Mdown & $\beta$ & 0.83 & 1500 & 45 & 7.7\% \\
Mdownbar & $\beta$ & -0.83 & 1500 & 45 & 7.7\% \\  
Mpassp & Vt & 0.17 & 500 & 45 & 16.7mV \\ 
Mpassp & $\beta$ & 0.17 & 500 & 45 & 8\% \\ 
Mpassbarp & Vt & -0.17 & 500 & 45 & 16.7mV \\
Mpassbarp & $\beta$ & -0.17 & 500 & 45 & 8\% \\ 
Mtop & $\beta$ & 0.13 & 900 & 45 & 6.0\% \\ 
Mtop & Vt & 0.10 & 900 & 45 & 12.4mV \\ 
Mbottom & $\beta$ & -0.067 & 500 & 45 & 13.3\% \\ 
Mbottom & Vt & 0.033 & 500 & 45 & 18.8mV \\ 
\hline 
\hline & $\sigma_{Voffset}$: & 9.6125mV & & & \\
\hline
\end{tabular} 
\caption[Sensitiviteitsanalyse van de SA in het finale geheugen]{Sensitiviteitsanalyse van de SA in het finale geheugen, er is overlap tussen de controlesignalen}
\label{tab:ourSA-sensanalysis-overlap}
\end{table}

\section{Paretosimulatie}
In het beginstadium van een ontwerp is nog niet duidelijk wat de impedantie aan de BL wordt. Het is deze impedantie die bepaalt wat het spanningsverschil is tussen het datasignaal en het referentiesignaal aan de sense amplifier. Bovendien kan het zijn dat er midden in het ontwerp besloten wordt om een andere impedantie te kiezen om alsnog te optimaliseren naar een andere variabele.
Natuurlijk is het mogelijk om één SA te gebruiken die voor elke impedantie een correcte en snelle werking zou garanderen. Dit zou echter een verspilling zijn van energie. In deze sectie wordt een pareto-oppervlak opgesteld waarbij er voor elk spanningsverschil de snelste en energiezuinigste SA-ontwerpen worden gekozen.

\subsection{Opstelling}
Uit een verzameling van allerhande SA [dit zijn drain-input latch-type SA waarvan de transistoren verschillend geschaald zijn - differentiële paren hebben zelfde afmetingen] worden enkel de pareto-optimale SA uitgekozen. De pareto-criteria zijn $\Delta V$, snelheid en dynamische energie. 

Voor deze opstelling worden de passgates weggelaten van de SA [dit is geoorloofd zoals bleek uit de sensitiviteitsanalyse], de last aan de ingangs-uitgangsknopen is een simpele CMOS inverter. De knopen zijn voorgeladen op 2 spanningen: 0,4V en 0,4V - $\Delta V$. Na 0,5ns wordt de SA aangezet en wordt de tijd gemeten tot wanneer de ingangs-uitgangsknopen geladen of ontladen zijn tot 99,9\% van hun finale waarde (VDD of VSS). Dit is wellicht een te strenge methode om de snelheid van de SA te bepalen aangezien de inverters al eerder zullen schakelen. Indien de snelheid van de 2 knopen verschilt, zal de traagste tijd genomen worden. De dynamische energie wordt opgemeten van het moment dat de SA wordt aangeschakeld tot dit tijdstip. Ook het statisch vermogen van de SA wordt opgemeten vanaf wanneer de ingangs-uitgangsknopen VDD en VSS bereiken. Uiteraard wordt ook geverifieerd of de SA wel correct heeft gelatcht.

Per sense amplifier worden er 250 Monte Carlo simulaties uitgevoerd met deze opstelling. Indien de SA niet elke keer correct functioneerde, wordt de SA verworpen. Latchte de SA wel elke keer correct, wordt het gemiddelde van de delay, dynamische energie  en statisch vermogen opgeslagen. Merk op dat de absolute waardes die opgeslagen worden niet zo nuttig zijn. Voor een praktisch ontwerp is de worst-case delay van de SA belangrijk, niet zozeer de gemiddelde. Deze gemiddelde waardes geven wel een beeld van hoe de SAs zich onderling meten.  

\subsection{Resultaten}
Op figuur \ref{fig:pareto} zijn de pareto-optimale resultaten getoond van de groep sense amplifiers. Het doel van deze simulatie is veeleer om de transistorafmetingen te situeren in functie van deze optimalizatievariabelen. Voor deze simulatieopstelling kan men enkel zeggen dat de kans dat de offsetspanning lager is dan $\Delta V$ minstens $1 -\frac{1}{250}$ is. Dit is een veel te kleine garantie voor een sense amplifier die misschien wel miljoenen keren zal gefabriceerd worden. Voor meer informatie over de verdeling van de offsetspanning te krijgen moet de standaardafwijking berekend worden met de sensitiviteitsanalyse.

\begin{figure}
  \centering
  \includegraphics[width=0.8\textwidth]{../fig/hfdstk-sensamp-pareto2.png}
  \caption[De pareto-optimale sense amplifiers]{De pareto-optimale sense amplifiers}
  \label{fig:pareto}
\end{figure}

\section{Besluit}
In dit hoofdstuk werd dieper ingegaan op de sense amplifier, die het kleine spanningsverschil tussen datasignaal en referentiesignaal correct moet versterken tot VDD en VSS. De belangrijkste eigenschap van de SA is de offsetspanning welke het gevolg is van variaties op transistorafmetingen en -karakteristieken. Deze kan voldoende klein gemaakt worden door de transistoren voldoende op te schalen. De offsetspanning kan statistisch beschreven worden met behulp van een sensitiviteitsanalyse. Tenslotte worden er ook uit een grote groep SA de pareto-optimale gekozen. De resultaten geven een idee van de grootteordes van transistorafmetingen voor een bepaalde offetspanning, snelheid en dynamische energie.
