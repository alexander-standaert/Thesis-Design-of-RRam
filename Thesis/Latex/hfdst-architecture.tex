\chapter{Geheugenarchitectuur}
\label{architectuur}
De afzonderlijke geheugencellen zullen uiteindelijk samengebracht moeten worden in een geheel.
In dit hoofdstuk zal de algemene structuur besproken worden alsook de vrijheidsgraden die in hoofdstuk \ref{optimalisatie} onderzocht worden voor een optimaal werkend systeem. Ten slotte zullen ook nog de bouwblokken aangekaard worden die meer uitvoerig besproken worden in de volgende hoofdstukken. 

\section{Van transistorniveau tot systeemniveau}
Hoewel het hele systeem op de chip in weze bestaat uit transistoren en passieve componenten, zal dit nog onmogelijk te begrijpen vallen op zulke grote schaal.
Het is daarom aangewezen om meer abstractie te maken en de componenten vanop een hoger niveau te bekijken.

\subsection{Cel}
%Een tekst staat nooit alleen. Dit wil zeggen dat er zeker ook referenties
%nodig zijn. Dit kan zowel naar on-line documenten\cite{wiki} als naar
%boeken\cite{pratchett06:_good_omens}.
Zoals besproken in hoofdstuk \ref{cel} zal dit het bouwblok zijn dat het vaakst terug zal te vinden zijn op het geheugensysteem.
De cel bestaat uit een memristor en een transistor. De geheugencel heeft drie terminals: de gate van de transistor, die zal verbonden worden met een wordline, de source van de transistor, die zal verbonden worden met een sourceline en tenslotte de terminal van de memristor, die zal verbonden worden met een bitline.
Een cel wiens memristor zich in een willekeurige resistieve staat bevindt is een datacel, terwijl de memristor van referentiecellen in een voorgeprogrammeerde en dus gekende resistieve staat verkeert.

\subsection{Branch}
In een branch worden er een bepaald aantal datacellen verbonden aan één BL en één SL. Dit aantal wordt \emph{Number of WL per Branch} (NoWLpB) genoemd en is een van de vrijheidsgraden van de geheugenarchitectuur. Naast alle datacellen is er ook nog één referentiecel verbonden aan de BL en SL van de branch.
Elke BL wordt via een pMOS-transistor verbonden aan de voedingspanning Vdd en via een nMOS-transistor aan de grondspanning Vss. In dit werk is er enkel een nMOS-transistor die de SL verbindt met Vss.\footnote{In een volledig geheugensysteem zou de SL via een pMOS-transistor ook nog verbonden zijn met een niet onderzochte spanningsknoop Vdd\_write. De pMOS zou dan worden aangezet voor schrijfwerking.}

\subsection{Local Block}
Verschillende BLs en SLs worden samengebracht in een Local Block, wiens vrijheidsgraad \emph{Number of Bit Lines per Local Block} (NoBLpLB) heet. In een LB bevinden er zich dus NoBLpLB*NoWLpB datacellen en NoBLpLB referentiecellen. Ook zitten er in een Local Block zowel BL- als WL-decoders.
%\section{Figuren}
%Figuren worden gebruikt om illustraties toe te voegen. Dit is dan ook de
%manier om beeldmateriaal toe te voegen zoals getoond wordt in
%figuur~\ref{fig:logo}.
%
%\begin{figure}
%  \centering
%  \includegraphics{logokul}
%  \caption{Het KU~Leuven logo.}
%  \label{fig:logo}
%\end{figure}
%
%\section{Tabellen}
%Tabellen kunnen gebruikt worden om informatie op een overzichtelijke te
%groeperen. Een tabel is echter geen rekenblad! Vergelijk maar eens
%tabel~\ref{tab:verkeerd} en tabel~\ref{tab:juist}. Welke tabel vind jij het
%duidelijkst?
%
%\begin{table}
%  \centering
%  \begin{tabular}{||l|lr||} \hline
%    gnats     & gram      & \$13.65 \\ \cline{2-3}
%              & each      & .01 \\ \hline
%    gnu       & stuffed   & 92.50 \\ \cline{1-1} \cline{3-3}
%    emu       &           & 33.33 \\ \hline
%    armadillo & frozen    & 8.99 \\ \hline
%  \end{tabular}
%  \caption{Een tabel zoals het niet moet.}
%  \label{tab:verkeerd}
%\end{table}
%
%\begin{table}
%  \centering
%  \begin{tabular}{@{}llr@{}} \toprule
%    \multicolumn{2}{c}{Item} \\ \cmidrule(r){1-2}
%    Animal    & Description & Price (\$)\\ \midrule
%    Gnat      & per gram    & 13.65 \\
%              & each        & 0.01 \\
%    Gnu       & stuffed     & 92.50 \\
%    Emu       & stuffed     & 33.33 \\
%    Armadillo & frozen      & 8.99 \\ \bottomrule
%  \end{tabular}
%  \caption{Een tabel zoals het beter is.}
%  \label{tab:juist}
%\end{table}

\section{Lorem ipsum}
Tenslotte gaan we hier nog wat tekst voorzien zodat er minstens een
bijkomende bladzijde aangemaakt wordt. Dat geeft de gelegenheid om eens te
zien hoe de koptekst en de voettekst zich gedragen.

\subsection{Lorem ipsum dolor sit amet, consectetur adipiscing elit}
Sed nec tortor id felis tristique sodales. Nulla nec massa eu dui fermentum
tincidunt. Integer ullamcorper ante eget eros posuere faucibus. Nam id
ligula ut augue pulvinar vulputate id at purus. Aenean condimentum tortor
eu mi placerat eget eleifend massa mollis. Nam est mi, sagittis quis
euismod eget, sagittis in nibh. Proin elit turpis, aliquam et imperdiet
sed, volutpat eu turpis.

Pellentesque vel enim tellus, vitae egestas turpis. Praesent malesuada elit
non nisi sollicitudin non blandit lacus tincidunt. Morbi blandit urna at
lectus ornare laoreet. Suspendisse turpis diam, lobortis dictum luctus
quis, commodo at lorem. Integer lacinia convallis ultricies. Sed quis augue
neque, eu malesuada arcu. Nullam vehicula, purus vitae sagittis pulvinar,
erat eros semper massa, eu egestas nibh erat quis magna. Cras pellentesque,
nisl eu dapibus volutpat, urna augue ornare quam, quis egestas lectus nulla
a lectus.

Vivamus dictum libero in massa cursus sed vulputate eros imperdiet. Donec
lacinia, libero ac lobortis egestas, nibh dui ornare arcu, luctus porttitor
velit massa sit amet quam. Maecenas scelerisque laoreet diam, vitae congue
quam adipiscing vitae. Aliquam cursus nisl a leo convallis eleifend
fermentum massa porta. Nunc libero quam, dapibus dapibus molestie sit amet,
faucibus vel nunc.

\subsection{Praesent auctor venenatis posuere}
Sed tellus augue, molestie in pulvinar lacinia, dapibus non ipsum. Fusce
vitae mi vitae enim ullamcorper hendrerit eu malesuada est. Proin iaculis
ante sed nibh tincidunt vel interdum libero posuere. Vivamus accumsan metus
quis felis congue suscipit dapibus enim mattis. Fusce mattis tortor eget
ipsum interdum sagittis auctor id metus.

Integer diam lacus, pharetra sit amet tempor et, tristique non lorem.
Aenean auctor, nisi eu interdum fermentum, lectus massa adipiscing elit,
sed facilisis orci odio a lectus. Proin mi nibh, tempus quis porta a,
viverra quis enim. In sollicitudin egestas libero, quis viverra velit
molestie eget. Nulla rhoncus, dolor a mollis vestibulum, lacus elit semper
nisi, nec sollicitudin sem urna eu magna. Nunc sed est urna, euismod congue
mi.

\subsection{Cras vulputate ultricies venenatis}
Vivamus eros urna, sodales accumsan semper vel, lobortis sit amet mauris.
Etiam condimentum eleifend lorem, ullamcorper ornare lectus aliquet vitae.
Praesent massa enim, interdum sit amet semper et, venenatis ut elit.
Quisque faucibus, quam ac lacinia imperdiet, nulla neque elementum purus,
tempus rutrum justo massa porta sapien. Vestibulum ante ipsum primis in
faucibus orci luctus et ultrices posuere cubilia Curae; Sed ultrices
interdum mi, et rhoncus sapien rutrum sed.

Duis elit orci, molestie quis sollicitudin sed, convallis non ante.
Maecenas tincidunt condimentum justo, et ultricies leo tristique vitae.
Vestibulum quis quam non lectus dapibus eleifend a vitae nibh. Nam nibh
justo, pharetra quis iaculis consequat, elementum quis justo. Etiam mollis
lacinia lacus, nec sollicitudin urna lobortis ac. Nulla facilisi.

Proin placerat risus eleifend erat ultricies placerat. Etiam rutrum magna
nec turpis euismod consectetur. Phasellus tortor odio, lacinia imperdiet
condimentum sed, faucibus commodo erat. Phasellus sed felis id ante
placerat ultrices. Aenean tempor justo in tortor volutpat eu auctor dolor
mollis. Aenean sit amet risus urna. Morbi viverra vehicula cursus.

\subsection{Donec nibh ante, consectetur et posuere id, tempus nec arcu}
Curabitur a tellus aliquet ipsum pellentesque scelerisque. Etiam congue,
risus et volutpat rutrum, est purus dapibus leo, non cursus metus felis
eget ligula. Vivamus facilisis tristique turpis, ut pretium lectus luctus
eleifend. Fusce magna sapien, ullamcorper vitae fringilla id, euismod quis
ante.

Phasellus volutpat, nunc et pharetra semper, sem justo adipiscing mauris,
id blandit magna quam et orci. Vestibulum a erat purus, ut molestie ante.
Vestibulum ante ipsum primis in faucibus orci luctus et ultrices posuere
cubilia Curae; Proin turpis diam, consequat ut ullamcorper ut, consequat eu
orci. Sed metus risus, fringilla nec interdum vel, interdum eu nunc.
Suspendisse vel sapien orci.

\subsection{Morbi et mauris tempus purus ornare vehicula}
Mauris sit amet diam quam, eget luctus purus. Sed faucibus, risus semper
eleifend iaculis, mi turpis bibendum nisl, quis cursus nibh nisl sit amet
ipsum. Vestibulum tempor urna vitae mi auctor malesuada eget non ligula.
Nullam convallis, diam vel ultrices auctor, eros eros egestas elit, sed
accumsan arcu tortor eget leo. Vestibulum orci purus, porttitor in pharetra
eget, tincidunt eget nisl. Nullam sit amet nulla dui, facilisis vestibulum
dui.

Donec faucibus facilisis mauris ac cursus. Duis rhoncus quam sed nisi
laoreet eu scelerisque massa tincidunt. Vivamus sit amet libero nec arcu
imperdiet tempor quis non libero. Sed consequat dignissim justo. Phasellus
ullamcorper, velit quis posuere vulputate, felis erat tincidunt mauris, at
vestibulum justo lectus et turpis. Maecenas lacinia convallis euismod.
Quisque egestas fermentum sapien eu dictum. Sed nec lacus in purus dictum
consequat quis vel nisl. Fusce non urna sem. Curabitur eu diam vitae elit
accumsan blandit. Nullam fermentum nunc et leo dictum laoreet. Donec semper
varius velit vel fringilla. Vivamus eu orci nunc.

\section{Besluit van dit hoofdstuk}
Als je in dit hoofdstuk tot belangrijke resultaten of besluiten gekomen
bent, dan is het ook logisch om het hoofdstuk af te ronden met een
overzicht ervan. Voor hoofdstukken zoals de inleiding en het
literatuuroverzicht is dit niet strikt nodig.

%%% Local Variables: 
%%% mode: latex
%%% TeX-master: "masterproef"
%%% End: 
