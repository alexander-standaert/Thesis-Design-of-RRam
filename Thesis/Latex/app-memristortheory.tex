\chapter{Leon Chua's memristortheorie}
\label{Chua}
In 1971 publiceerde Leon Chua, een Amerikaans onderzoeker in o.a. niet-lineaire circuittheorie, een artikel waarin hij opmerkte dat er voor de 4 fundamentele circuitvariabelen (de spanning v, stroom i, lading q en flux $\phi$\footnote{$\phi(t) =  \int^{t}_{-\infty} v(\tau) \, d\tau $, voor een ideale inductantie is dit hetzelfde als magnetische flux}) van 6 mogelijke onderlinge relaties er slechts 5 gekend waren:\\\\ $q(t) =  \int^{t}_{-\infty} i(\tau) \, d\tau $\\ $\phi(t) =  \int^{t}_{-\infty} v(\tau) \, d\tau $\\ $v(t)=R*i(t)$\\ $q(t)=C*v(t)$ \\$\phi(t) = L*i(t)$ \\\\ volgen uit de wetten van Maxwell en uit de definities van de weerstand, spoel en condensator, maar er ontbrak dus nog een relatie tussen $\phi$ en q\cite{Chu71}. Hij suggereerde dat er een 4e nog niet ontdekte passieve 2-pool moest bestaan die dit verband herbergde. Hij stelde dat $M(q)= \frac{d\phi(q)}{dq}$ met M de \emph{memristance}.
Hieruit volgt dat voor dit element $v(t)=M(q(t)) i(t)$.\\ Indien er een lineair verband bestaat tussen $\phi$ en q, gedraagt dit element zich als een gewone weerstand. Enkel wanneer er een niet-lineair verband bestaat, beginnen er zich interessante fenomenen voor te doen. Zo gedraagt het element zich ogenblikkelijk als een weerstand, maar gaat deze weerstandswaarde variëren in de tijd aan de hand van de stroom die er doorgelopen heeft.\\
Gebaseerd op deze conclusie doopte hij deze component de memristor, een contractie van memory en resistor.
Chua beëindigde zijn artikel met te erkennen dat er op dat moment nog geen fysische memristor was ontdekt, maar dat dit in de toekomst wel kon gebeuren, al dan niet zelfs per ongeluk. Hij gaf aan dat er misschien al in die tijd materialen met memristorkarakteristieken gebruikt werden, maar dat men hier over keek. Hij zou gelijk krijgen (min of meer).

