\chapter{Besluit}
\label{besluit}
In dit werk werd een RRAM leescircuit ontworpen. De 1T1R-cel die de informatie bevat bestaat uit een minimale transistor en een resistief geheugenelement, de memristor (hoofdstuk \ref{cell}). Voor simulaties werden de geheugenelementen voorgesteld als weerstanden waarvan de impedantie gebaseerd is op hafniumoxidememristoren. Verschillende cellen worden in geheugenmatrices gegroepeerd (hoofstuk \ref{architecture}), aan deze matrices worden decoders en passgates toegevoegd die samen een local block vormen. Een local block kan aan diens uitgang zowel een datasignaal leveren als een referentiesignaal. De uitgangen van 2 LBs worden aan de ingangs-uitgangsknopen van een sense amplifier gehangen; de combinatie van twee LBs en SA heet global block. Data- en referentiesignalen worden opgewekt door een spanningsdeling met een bepaalde lastimpedantie en celimpedantie. Voor datasignaal vindt er een spanningsdeling plaats op één BL, deze BL-spanning wordt naar de uitgang van het LB overgebracht door de bijhorende passgate te activeren. Voor het referentiesignaal worden er op meerdere BLs spanningsdelingen uitgevoerd, de referentiespanning wordt gevormd door de BLs kort te sluiten aan de uitgang met de passgates. Door meerdere referentiecellen te gebruiken kan de distributie van het referentiesignaal gemanipuleerd worden. Voor een zo groot mogelijk verschil tussen data- en referentiespanning blijkt er een optimale lastimpedantie (hoofdstuk \ref{loadanalysis}). Verschillende topologieën impedantie zijn onderzocht naar BL-spanningsverschil, snelheid en spanningsval over geheugenelement, alsook de invloed van variabiliteit hierop. Omwille van dit laatste is het niet mogelijk om met transistoren met minimale lengtes te werken. Een enkele transistor met niet-minimale afmetingen blijkt er als beste uit te komen. De sense amplifier moet het kleine spanningsverschil tussen data- en referentiesignaal correct versterken tot de voedingspanning (hoofdstuk \ref{sensamp}). De gebruikte topologie in dit werk is de drain-input latch-type SA. De belangrijkste eigenschap van een SA wat correcte werking betreft is de offsetspanning. De distributie hiervan wordt in kaart gebracht met sensitiviteitsanalyses. Door korte overlap tussen passgate- en SA-enable-signaal kan de spreiding kleiner gemaakt worden. Het RC-latch-effect zorgt ervoor dat de SA even geen invloed merkt van de grote BL-capaciteit, waaraan die blootgesteld wordt tijdens de overlap. Op basis van een lineaire sweep van transistorafmetingen werden  pareto-optimale SAs gekozen wat snelheid, dynamische energie en offsetspanning betreft. Naast lastimpedantie en SAs is er in het geheugen ook nood aan omringende logica zoals buffers, passgates, decoders,... Deze werden onderzocht in hoofdstuk \ref{periphery}. Hoofdstuk \ref{timing-optimization} brengt de timing van alle signalen in het geheugen in kaart. Hierbij werd er gekeken welke beperkingen er opgelegd moeten worden aan de architectuur om een juiste timing te hebben. Met deze kennis worden een aantal geheugens ontworpen van 4Mbit en met elkaar vergeleken op het vlak van snelheid, energieverbruik en oppervlaktegebruik. Uiteindelijk wordt een geheugenarchitectuur met 32WLs, 32BLs en 512GBs gekozen als eindontwerp. Hierop wordt er een speed-vdd-test (hoofdstuk \ref{final}) uitgevoerd en wordt de prestatie van het circuit vergeleken met de literatuur. Men kan besluiten dat met de afbakening in het achterhoofd de schakeling een goede prestatie levert t.o.v. schakelingen in de literatuur.
